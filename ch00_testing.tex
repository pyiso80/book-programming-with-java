\chapter{\myenchttl{Karel the Robot}  စက်ရုပ်ကားရဲလ်}
\begin{sloppypar}
    အသုံးပြုရတဲ့ \toem{\myemen{control flow statements} \myemmm{များတွင်} \myemcode{for, while, if, if else} \myemmm{တို့ပါဝင်သည်။} } \color{tungsten} တွေကို  ဒီအခန်းမှာ ‌လေ့လာကြမယ်။ ကမ္ဘာ့

    အသုံးပြုရတဲ့ \todef{\mydefen{control flow statements} \mydefmm{များတွင်} \mydefcode{for, while, if, if else} \mydefmm{တို့ပါဝင်သည်။} } \color{tungsten} တွေကို  ဒီအခန်းမှာ ‌လေ့လာကြမယ်။ ကမ္ဘာ့

\end{sloppypar}

\section{\myensecttl{Commands Karel Understands} ကားရဲလ်နားလည်သော ကွန်မန်းများ }


\subsection{\myensubsecttl{Commands Karel Understands} ကားရဲလ်နားလည်သော ကွန်မန်းများ \mycodesubsecttl{for, while if}}
\subsubsection{\myensubsubsecttl{Commands Karel Understands} ကားရဲလ်နားလည်သော ကွန်မန်းများ \mycodesubsubsecttl{for, while if}}



\begin{figure}
\centering
\begin{subfigure}{0.4\textwidth}
    \includegraphics[width=\textwidth]{example-image}
    \caption{Firts subfigure.}
    \label{fig:first}
\end{subfigure}
\hfill
\begin{subfigure}{0.4\textwidth}
    \includegraphics[width=\textwidth]{example-image}
    \caption{Second subfigure.}
    \label{fig:second}
\end{subfigure}
\hfill
\begin{subfigure}{0.4\textwidth}
    \includegraphics[width=\textwidth]{example-image}
    \caption{Third subfigure.}
    \label{fig:third}
\end{subfigure}
        
\caption{Creating subfigures in \LaTeX.}
\label{fig:figures}
\end{figure}

\begin{figure}
    \adjustimage{width=.6\textwidth,left}{example-image}
    \caption{centered image}
\end{figure}

% or even shorter
\noindent\adjustimage{width=.6\textwidth,left,caption={your caption},label={some label},figure}{example-image}

\afterpage{\blankpage}
