\chapter{Control Flow}
\XeTeXlinebreaklocale "my_MM"  %Myanmar line and character breaks
\XeTeXinterwordspaceshaping=2
\begin{sloppypar}

\mmcommand တွေကို ထပ်ခါထပ်ခါ ပြန်ကျော့ခိုင်းဖို့ လိုအပ်လေ့ရှိတယ်။ ကားရဲလ်ကို ရှေ့ကို ၂၅ လှမ်း ရွှေ့ခိုင်းချင်တယ် ဆိုပါစို့။ \txtmycode{move(); move(); move(); ...} (၂၅) ခါ ရေးလို့‌တော့ရတာပေါ့။ ဒါပေမယ့် စာရိုက်ရတာ အချိန်လည်းကုန် လက်လဲညောင်း ဖြစ်မယ်။ \txtmycode{move();} ကို ၂၅ ခါကျော့ပေးပါလို့ ခိုင်းလို့ရရင် ပိုမကောင်းဘူးလား။ 

ကားရဲလ်ရဲ့ ရှေ့တည့်တည့် ခပ်လှမ်းလှမ်းမှာ နံရံတစ်ခုရှိနေမယ်။ ဘယ်လောက်ဝေးလဲ မသိဘူးဆိုပါစို့။ ကားရဲလ်ကို နံရံဆီရောက်အောင် သွားခိုင်းချင်တယ်။ နံရံက ဘယ်လောက်ဝေးလဲ မသိတော့ \txtmycode{move} ကို ဘယ်နှစ်ကြိမ် ကျော့ခိုင်းရမလဲ မသိနိုင်ဘူး။ ရှေ့မှာရှင်းနေသေးသ၍ \txtmycode{move} ပါလို့သာ ခိုင်းလို့ရမယ်ဆိုရင် တော်တော်လေး အဆင်ပြေပြီ။ 

ရှေ့မှာပြောခဲ့တဲ့ နံရံအောက််ခြေမှာ \mmbeeper တစ်ခု ရှိနေနိုင်တယ်။ ရှိချင်မှလဲ ရှိမယ်ဆိုပါစို့။ \mmbeeper ရှိနေခဲ့ရင် နံရံ အခြားတဘက်မှာ \mmbeeper ကို ထားခိုင်းချင်တယ် ဆိုပါစို့။  ဒါဆိုရင် \mmbeeper ရှိနေမှပဲ ကားရဲလ်ကို အခြားတဘက်ကို ရွှေ့ဖို့လ။ \mmbeeper မရှိဘူးဆိုရင် အဲဒီ \mmcommand တွေကို မလုပ်ဆောင်စေချင်ဘူး။  

ကားရဲလ်ကို အခြေအနေတခု မှန်တော့မှပဲ \mmcommand တွေကို လုပ်ဆောင်စေချင်တာမျိုးလဲ ရှိတယ်။ အဲဒီ အခြေနေနဲ့ မကိုက်ညီဘူး (တနည်းအားဖြင့် အခြေအနေက မှားနေခဲ့ရင်) \mmcommand တွေကို မလုပ်ဆောင်ပဲ ကျော်သွားစေချင်တယ်။ နောက်ထပ်တမျိုးက အခြေအနေက မှန်ခဲ့ရင် 




ပရိုဂရမ်းမင်း\txtmyparaen{(programming)} ဆိုတာဘာလဲ။ ဒါကိုဖြေဖို့ စာတွေအများကြီးရေးပြီး ရှင်းပြလို့ရပေမယ့် အများကြီးသိပ်မပြောပဲ လက်တွေ့ \txtmyparaen{program} လေးတွေစရေးကြည့်ပြီးမှ ပရိုဂရမ်းမင်းဆိုတာ ဘာလဲ ပြောပြလိုက်ရင် ပိုပြီးနားလည်ရလွယ်တယ်။ ဒါ့ကြောင့် စက်ရုပ်ကားရဲလ်\txtmyparaen{(Karel the Robot)}ကို အလုပ်တွေ လုပ်ခိုင်းဖို့ ပရိုဂရမ်\txtmyparaen{(program)} လေးတွေ အရင်ဆုံး ရေးကြည့်ကြမယ်။
\section{{\myttlcode{for}} loop ကိုအသုံးပြုခြင်း}
စက်ရုပ်ကားရဲလ်က အခြေခံအားဖြင့် \txtmycode{move, turnLeft, putBeeper, pickBeeper} ဆိုတဲ့ ကွန်မန်း\txtmyparaen{(command)} လေးခုကို နားလည်တယ်။ \txtmycode{move} ကွန်မန်းက သူရပ်နေတဲ့ ကွန်နာ\txtmyparaen{(corner)}ကနေ ရှေ့တည့်တည့် ကပ်ရပ် ကွန်နာဆီကို ရွှေ့ခိုင်းတာ။ ပုံမှာပြထားတဲ့ ကြက်ခြေခတ်လေးတွေကို ကွန်နာလို့ခေါ်တယ်။ 

\txtmycode{turnLeft} ကွန်မန်းကတော့ ဘယ်ဘက်လှည့်ခိုင်းတာ။ ကားရဲမျက်နှာမူရာ အရပ်မျက်နှာ တစ်ခုချင်းစီအတွက် \txtmycode{turnLeft} ကွန်မန်းပေးလိုက်တဲ့ အခါမှာ ဘယ်အရပ်ကိုလှည့်သွားမလဲ ပုံမှာတွေ့နိုင်တယ်။
\section{{\myttlcode{for}} loop ကိုအသုံးပြုခြင်း}
ကားရဲလ်ကို \txtmycode{putBeeper} ကွန်မန်းပေးလိုက်ရင်တော့ ကားရဲလ်က သူရှိနေတဲ့ ကွန်နာမှာ ဘိပါ\txtmyparaen{(beeper)} လို့ခေါ်တဲ့ အတုံးလေး တစ်ခုချထားလိမ့်မယ်။ ဒီကွန်မန်း မပေးရသေးခင်နဲ့ ပေးပြီးအခြေအနေကို ပုံမှာယှဉ်ပြထားတယ်။ ကွန်မန်းပေးပြီးသွားတဲ့အခါ စိန်တုံးပုံစံ ဘိပါတုံးလေး ရှိနေတာ တွေ့ရမယ်။ 
\section{{\myttlcode{for}} loop ကိုအသုံးပြုခြင်း}

\txtmycode{pickBeeper} ကွန်မန်းက ဘိပါကောက်ခိုင်းတာပါ။ ကားရဲလ်ရောက်နေတဲ့ ကွန်နာမှာ ဘိပါရှိရင် ဒီကွန်မန်းနဲ့ ကောက်ခိုင်းလို့ရတယ်။ ကွန်နာတစ်ခုမှာ ဘိပါတစ်ခုမက ရှိနိုင်တယ်။ တစ်ခုမှ မရှိတာလဲ ဖြစ်နိုင်တယ်။ ပုံမှာဆိုရင် ကွန်နာတစ်ခုမှာပဲ ဘိပါရှိနေပြီး ကျန်တဲ့ ကွန်နာတွေမှာက ဘိပါမရှိပါဘူး။

ပရိုဂရမ်းမင်း\txtmyparaen{(programming)} ဆိုတာဘာလဲ။ ဒါကိုဖြေဖို့ စာတွေအများကြီးရေးပြီး ရှင်းပြလို့ရသလို အများကြီးသိပ်မပြောပဲ လက်တွေ့ \txtmyparaen{program} လေးတွေစရေးကြည့်ပြီးမှ ပရိုဂရမ်းမင်းဆိုတာ ဘာလဲ ပြောပြလိုက်ရင် ပိုပြီးနားလည်ရလွယ်တယ်။ ဒါ့ကြောင့် စက်ရုပ်ကားရဲလ်\txtmyparaen{(Karel the Robot)}ကို အလုပ်တွေ လုပ်ခိုင်းဖို့ ပရိုဂရမ်\txtmyparaen{(program)} လေးတွေ အရင်ဆုံး ရေးကြည့်ကြမယ်။

စက်ရုပ်ကားရဲလ်က အခြေခံအားဖြင့် \txtmycode{move, turnLeft, putBeeper, pickBeeper} ဆိုတဲ့ ကွန်မန်း\txtmyparaen{(command)} လေးခုကို နားလည်တယ်။ \txtmycode{move} ကွန်မန်းက သူရပ်နေတဲ့ ကွန်နာ\txtmyparaen{(corner)}ကနေ ရှေ့တည့်တည့် ကပ်ရပ် ကွန်နာဆီကို ရွှေ့ခိုင်းတာ။ ပုံမှာပြထားတဲ့ ကြက်ခြေခတ်လေးတွေကို ကွန်နာလို့ခေါ်တယ်။ 

\txtmycode{turnLeft} ကွန်မန်းကတော့ ဘယ်ဘက်လှည့်ခိုင်းတာ။ ကားရဲမျက်နှာမူရာ အရပ်မျက်နှာ တစ်ခုချင်းစီအတွက် \txtmycode{turnLeft} ကွန်မန်းပေးလိုက်တဲ့ အခါမှာ ဘယ်အရပ်ကိုလှည့်သွားမလဲ ပုံမှာတွေ့နိုင်တယ်။

ကားရဲလ်ကို \txtmycode{putBeeper} ကွန်မန်းပေးလိုက်ရင်တော့ ကားရဲလ်က သူရှိနေတဲ့ ကွန်နာမှာ ဘိပါ\txtmyparaen{(beeper)} လို့ခေါ်တဲ့ အတုံးလေး တစ်ခုချထားလိမ့်မယ်။ ဒီကွန်မန်း မပေးရသေးခင်နဲ့ ပေးပြီးအခြေအနေကို ပုံမှာယှဉ်ပြထားတယ်။ ကွန်မန်းပေးပြီးသွားတဲ့အခါ စိန်တုံးပုံစံ ဘိပါတုံးလေး ရှိနေတာ တွေ့ရမယ်။ 

test 

\txtmycode{pickBeeper} ကွန်မန်းက ဘိပါကောက်ခိုင်းတာပါ။ ကားရဲလ်ရောက်နေတဲ့ ကွန်နာမှာ ဘိပါရှိရင် ဒီကွန်မန်းနဲ့ ကောက်ခိုင်းလို့ရတယ်။ ကွန်နာတစ်ခုမှာ ဘိပါတစ်ခုမက ရှိနိုင်တယ်။ တစ်ခုမှ မရှိတာလဲ ဖြစ်နိုင်တယ်။ ပုံမှာဆိုရင် ကွန်နာတစ်ခုမှာပဲ ဘိပါရှိနေပြီး ကျန်တဲ့ ကွန်နာတွေမှာက ဘိပါမရှိပါဘူး။

ပရိုဂရမ်းမင်း\txtmyparaen{(programming)} ဆိုတာဘာလဲ။ ဒါကိုဖြေဖို့ စာတွေအများကြီးရေးပြီး ရှင်းပြလို့ရသလို အများကြီးသိပ်မပြောပဲ လက်တွေ့ \txtmyparaen{program} လေးတွေစရေးကြည့်ပြီးမှ ပရိုဂရမ်းမင်းဆိုတာ ဘာလဲ ပြောပြလိုက်ရင် ပိုပြီးနားလည်ရလွယ်တယ်။ ဒါ့ကြောင့် စက်ရုပ်ကားရဲလ်\txtmyparaen{(Karel the Robot)}ကို အလုပ်တွေ လုပ်ခိုင်းဖို့ ပရိုဂရမ်\txtmyparaen{(program)} လေးတွေ အရင်ဆုံး ရေးကြည့်ကြမယ်။
\section{{\myttlcode{for}} loop ကိုအသုံးပြုခြင်း}

စက်ရုပ်ကားရဲလ်က အခြေခံအားဖြင့် \txtmycode{move, turnLeft, putBeeper, pickBeeper} ဆိုတဲ့ ကွန်မန်း\txtmyparaen{(command)} လေးခုကို နားလည်တယ်။ \txtmycode{move} ကွန်မန်းက သူရပ်နေတဲ့ ကွန်နာ\txtmyparaen{(corner)}ကနေ ရှေ့တည့်တည့် ကပ်ရပ် ကွန်နာဆီကို ရွှေ့ခိုင်းတာ။ ပုံမှာပြထားတဲ့ ကြက်ခြေခတ်လေးတွေကို ကွန်နာလို့ခေါ်တယ်။ 

\txtmycode{turnLeft} ကွန်မန်းကတော့ ဘယ်ဘက်လှည့်ခိုင်းတာ။ ကားရဲမျက်နှာမူရာ အရပ်မျက်နှာ တစ်ခုချင်းစီအတွက် \txtmycode{turnLeft} ကွန်မန်းပေးလိုက်တဲ့ အခါမှာ ဘယ်အရပ်ကိုလှည့်သွားမလဲ ပုံမှာတွေ့နိုင်တယ်။

ကားရဲလ်ကို \txtmycode{putBeeper} ကွန်မန်းပေးလိုက်ရင်တော့ ကားရဲလ်က သူရှိနေတဲ့ ကွန်နာမှာ ဘိပါ\txtmyparaen{(beeper)} လို့ခေါ်တဲ့ အတုံးလေး တစ်ခုချထားလိမ့်မယ်။ ဒီကွန်မန်း မပေးရသေးခင်နဲ့ ပေးပြီးအခြေအနေကို ပုံမှာယှဉ်ပြထားတယ်။ ကွန်မန်းပေးပြီးသွားတဲ့အခါ စိန်တုံးပုံစံ ဘိပါတုံးလေး ရှိနေတာ တွေ့ရမယ်။ 

\txtmycode{pickBeeper} ကွန်မန်းက ဘိပါကောက်ခိုင်းတာပါ။ ကားရဲလ်ရောက်နေတဲ့ ကွန်နာမှာ ဘိပါရှိရင် ဒီကွန်မန်းနဲ့ ကောက်ခိုင်းလို့ရတယ်။ ကွန်နာတစ်ခုမှာ ဘိပါတစ်ခုမက ရှိနိုင်တယ်။ တစ်ခုမှ မရှိတာလဲ ဖြစ်နိုင်တယ်။ ပုံမှာဆိုရင် ကွန်နာတစ်ခုမှာပဲ ဘိပါရှိနေပြီး ကျန်တဲ့ ကွန်နာတွေမှာက ဘိပါမရှိပါဘူး။
\section{{\myttlcode{for}} loop ကိုအသုံးပြုခြင်း}

စက်ရုပ်ကားရဲလ်က အခြေခံအားဖြင့် \txtmycode{move, turnLeft, putBeeper, pickBeeper} ဆိုတဲ့ ကွန်မန်း\txtmyparaen{(command)} လေးခုကို နားလည်တယ်။ \txtmycode{move} ကွန်မန်းက သူရပ်နေတဲ့ ကွန်နာ\txtmyparaen{(corner)}ကနေ ရှေ့တည့်တည့် ကပ်ရပ် ကွန်နာဆီကို ရွှေ့ခိုင်းတာ။ ပုံမှာပြထားတဲ့ ကြက်ခြေခတ်လေးတွေကို ကွန်နာလို့ခေါ်တယ်။ 

\txtmycode{turnLeft} ကွန်မန်းကတော့ ဘယ်ဘက်လှည့်ခိုင်းတာ။ ကားရဲမျက်နှာမူရာ အရပ်မျက်နှာ တစ်ခုချင်းစီအတွက် \txtmycode{turnLeft} ကွန်မန်းပေးလိုက်တဲ့ အခါမှာ ဘယ်အရပ်ကိုလှည့်သွားမလဲ ပုံမှာတွေ့နိုင်တယ်။
\section{{\myttlcode{for}} loop ကိုအသုံးပြုခြင်း}

စက်ရုပ်ကားရဲလ်က အခြေခံအားဖြင့် \txtmycode{move, turnLeft, putBeeper, pickBeeper} ဆိုတဲ့ ကွန်မန်း\txtmyparaen{(command)} လေးခုကို နားလည်တယ်။ \txtmycode{move} ကွန်မန်းက သူရပ်နေတဲ့ ကွန်နာ\txtmyparaen{(corner)}ကနေ ရှေ့တည့်တည့် ကပ်ရပ် ကွန်နာဆီကို ရွှေ့ခိုင်းတာ။ ပုံမှာပြထားတဲ့ ကြက်ခြေခတ်လေးတွေကို ကွန်နာလို့ခေါ်တယ်။ 

\txtmycode{turnLeft} ကွန်မန်းကတော့ ဘယ်ဘက်လှည့်ခိုင်းတာ။ ကားရဲမျက်နှာမူရာ အရပ်မျက်နှာ တစ်ခုချင်းစီအတွက် \txtmycode{turnLeft} ကွန်မန်းပေးလိုက်တဲ့ အခါမှာ ဘယ်အရပ်ကိုလှည့်သွားမလဲ ပုံမှာတွေ့နိုင်တယ်။
\section{{\myttlcode{for}} loop ကိုအသုံးပြုခြင်း}

စက်ရုပ်ကားရဲလ်က အခြေခံအားဖြင့် \txtmycode{move, turnLeft, putBeeper, pickBeeper} ဆိုတဲ့ ကွန်မန်း\txtmyparaen{(command)} လေးခုကို နားလည်တယ်။ \txtmycode{move} ကွန်မန်းက သူရပ်နေတဲ့ ကွန်နာ\txtmyparaen{(corner)}ကနေ ရှေ့တည့်တည့် ကပ်ရပ် ကွန်နာဆီကို ရွှေ့ခိုင်းတာ။ ပုံမှာပြထားတဲ့ ကြက်ခြေခတ်လေးတွေကို ကွန်နာလို့ခေါ်တယ်။ 

\txtmycode{turnLeft} ကွန်မန်းကတော့ ဘယ်ဘက်လှည့်ခိုင်းတာ။ ကားရဲမျက်နှာမူရာ အရပ်မျက်နှာ တစ်ခုချင်းစီအတွက် \txtmycode{turnLeft} ကွန်မန်းပေးလိုက်တဲ့ အခါမှာ ဘယ်အရပ်ကိုလှည့်သွားမလဲ ပုံမှာတွေ့နိုင်တယ်။
\section{{\myttlcode{for}} loop ကိုအသုံးပြုခြင်း}

စက်ရုပ်ကားရဲလ်က အခြေခံအားဖြင့် \txtmycode{move, turnLeft, putBeeper, pickBeeper} ဆိုတဲ့ ကွန်မန်း\txtmyparaen{(command)} လေးခုကို နားလည်တယ်။ \txtmycode{move} ကွန်မန်းက သူရပ်နေတဲ့ ကွန်နာ\txtmyparaen{(corner)}ကနေ ရှေ့တည့်တည့် ကပ်ရပ် ကွန်နာဆီကို ရွှေ့ခိုင်းတာ။ ပုံမှာပြထားတဲ့ ကြက်ခြေခတ်လေးတွေကို ကွန်နာလို့ခေါ်တယ်။ 

\txtmycode{turnLeft} ကွန်မန်းကတော့ ဘယ်ဘက်လှည့်ခိုင်းတာ။ ကားရဲမျက်နှာမူရာ အရပ်မျက်နှာ တစ်ခုချင်းစီအတွက် \txtmycode{turnLeft} ကွန်မန်းပေးလိုက်တဲ့ အခါမှာ ဘယ်အရပ်ကိုလှည့်သွားမလဲ ပုံမှာတွေ့နိုင်တယ်။
\section{{\myttlcode{for}} loop ကိုအသုံးပြုခြင်း}

စက်ရုပ်ကားရဲလ်က အခြေခံအားဖြင့် \txtmycode{move, turnLeft, putBeeper, pickBeeper} ဆိုတဲ့ ကွန်မန်း\txtmyparaen{(command)} လေးခုကို နားလည်တယ်။ \txtmycode{move} ကွန်မန်းက သူရပ်နေတဲ့ ကွန်နာ\txtmyparaen{(corner)}ကနေ ရှေ့တည့်တည့် ကပ်ရပ် ကွန်နာဆီကို ရွှေ့ခိုင်းတာ။ ပုံမှာပြထားတဲ့ ကြက်ခြေခတ်လေးတွေကို ကွန်နာလို့ခေါ်တယ်။ 

\txtmycode{turnLeft} ကွန်မန်းကတော့ ဘယ်ဘက်လှည့်ခိုင်းတာ။ ကားရဲမျက်နှာမူရာ အရပ်မျက်နှာ တစ်ခုချင်းစီအတွက် \txtmycode{turnLeft} ကွန်မန်းပေးလိုက်တဲ့ အခါမှာ ဘယ်အရပ်ကိုလှည့်သွားမလဲ ပုံမှာတွေ့နိုင်တယ်။
\section{{\myttlcode{for}} loop ကိုအသုံးပြုခြင်း}

စက်ရုပ်ကားရဲလ်က အခြေခံအားဖြင့် \txtmycode{move, turnLeft, putBeeper, pickBeeper} ဆိုတဲ့ ကွန်မန်း\txtmyparaen{(command)} လေးခုကို နားလည်တယ်။ \txtmycode{move} ကွန်မန်းက သူရပ်နေတဲ့ ကွန်နာ\txtmyparaen{(corner)}ကနေ ရှေ့တည့်တည့် ကပ်ရပ် ကွန်နာဆီကို ရွှေ့ခိုင်းတာ။ ပုံမှာပြထားတဲ့ ကြက်ခြေခတ်လေးတွေကို ကွန်နာလို့ခေါ်တယ်။ 

\txtmycode{turnLeft} ကွန်မန်းကတော့ ဘယ်ဘက်လှည့်ခိုင်းတာ။ ကားရဲမျက်နှာမူရာ အရပ်မျက်နှာ တစ်ခုချင်းစီအတွက် \txtmycode{turnLeft} ကွန်မန်းပေးလိုက်တဲ့ အခါမှာ ဘယ်အရပ်ကိုလှည့်သွားမလဲ ပုံမှာတွေ့နိုင်တယ်။

\end{sloppypar}
