%!TEX TS-program = xelatex  
%!TEX encoding = UTF-8 Unicode
\documentclass[twoside, hidelinks]{book}
\hfuzz=10000pt
\hbadness=99999
\usepackage[paperheight=9.25in,
                paperwidth=7.38in,
                left=1.4in,
                right=.75in,
                top=.75in,
                bottom=1.25in,
                footskip=0.6in,
                heightrounded]{geometry}
                
\usepackage[dvipsnames]{xcolor}
\usepackage{varioref}
\usepackage{xspace}
\usepackage{fancyhdr}
\usepackage{graphicx}
\usepackage{blindtext}
\usepackage{fontspec}
\usepackage{titlesec} 
\usepackage{titletoc} 
\usepackage{listings}
\usepackage{caption}
\usepackage[unicode=true]{hyperref}


%%%  Replace english with Myanmar, tweaking varioref package 
\def\reftextfaceafter{on the \reftextvario{facing}{next} စာမျက်နှာ}%
\def\reftextfacebefore{on the \reftextvario{facing}{preceding}စာမျက်နှာ}%
\def\reftextafter{on the \reftextvario{following}{next} စာမျက်နှာ}%
\def\reftextbefore{on the \reftextvario{preceding}{previous} စာမျက်နှာ}%
\def\reftextcurrent{on \reftextvario{this}{the current} စာမျက်နှာ}%
\def\reftextfaraway#1{စာမျက်နှာ~\pageref{#1}}%
\def\reftextpagerange#1#2{on စာမျက်နှာ၂၂~\pageref{#1}--\pageref{#2}}%
\def\reftextlabelrange#1#2{\ref{#1} to~\ref{#2}}%


%%% Replace english names with Myanmar
\renewcommand{\contentsname}{မာတိကာ} 
\renewcommand{\chaptername}{}
\renewcommand{\figurename}{ပုံ}


%%% definition of Apple colors
\definecolor{iron}{HTML}{5E5E5E}
\definecolor{steel}{HTML}{797979}
\definecolor{tungsten}{HTML}{424242}
\definecolor{lead}{HTML}{212121}
\definecolor{steelblue}{HTML}{36648B}


%%% Page style related stuff
\fancyhf{}
\fancyhead[LE]{\color{steelblue}\mmpdkb\nouppercase{\leftmark}}
\fancyhead[RO]{\color{steelblue}\mmpdkb\nouppercase{\rightmark}}
\fancyfoot[LE,RO]{\thepage}
\pagestyle{fancy}
\setlength{\headsep}{15pt}
\setlength{\headheight}{30pt}% ...at least 51.60004pt
\renewcommand\headrule{
\begin{minipage}
    {1\textwidth}
    \color{MidnightBlue}\hrule width \hsize \kern 1mm 
    \color{MidnightBlue}\hrule width \hsize height 2pt 
\end{minipage}}%


%%% Font related tweaks
\defaultfontfeatures{Script=Myanmar,Mapping=tex-text} 
\fontspec[Script=Myanmar, BoldFont={Padauk Book Bold}, ]{Padauk Book}
\setmainfont[Color=tungsten, UprightFont={Padauk Book},
                BoldFont={Padauk Book Bold},
                ItalicFont={Padauk Book},
                BoldItalicFont={Padauk Book Bold},
                SmallCapsFont={Padauk Book},
                SlantedFont={Padauk Book}]{Padauk Book}
                [Renderer=Harfbuzz,Script=Myanmar]


%%% Define fonts english font to be used in between Myanmar text               
% Helvetica Neue, Geneva, Tahoma
% Font for normal english words and program code related words to be used in paragraph
\newfontfamily\myparaen[Color=tungsten, Scale=MatchLowercase]{Verdana}
\newfontfamily\mycode[Color=lead, Scale=1]{Ubuntu Mono}

\DeclareTextFontCommand{\txtmyparaen}{\myparaen}
\DeclareTextFontCommand{\txtmycode}{\mycode}

\DeclareCaptionFont{mysize}{\color{iron}\fontspec[Script=Myanmar,Scale=0.9]{Pyidaungsu Bold}\selectfont}
\captionsetup{font=mysize}

\newcommand{\mmem}[1]{\textbf{\color{blue}#1}}
\newcommand{\mmpdk}{\fontspec[Script=Myanmar,Scale=1]{Myanmar Tagu}\selectfont}
\newcommand{\mmpdkb}{\fontspec[Script=Myanmar,Scale=1]{Myanmar Tagu}\selectfont} 
\newcommand{\mmpdkbk}{\fontspec[Script=Myanmar,Scale=1]{Myanmar Tagu} \selectfont}
\newcommand{\mmpdkbkb}{\fontspec[Script=Myanmar,Scale=1]{Myanmar Tagu} \selectfont}
\newcommand{\myttlcode}{\fontspec[Scale=1]{Ubuntu Mono Bold} \selectfont} 


%changes the spacing for everything in the document, including footnotes and tables 
\renewcommand{\baselinestretch}{1.35} 


%%% Auto numbering in Myanmar %%%
%This macro is to produce myanmar numbering by adopting the thai numbering method
\makeatletter  
\def\@mmnum#1{\expandafter\@@mmnum\number#1\@nil}  
\def\@@mmnum#1{%  
  \ifx#1\@nil  
  \else  
  \char\numexpr#1+"1040\relax  
  \expandafter\@@mmnum\fi  
}  
\renewcommand\@arabic{\@mmnum} % to reset number in \arabic to \mmnum 
\makeatother 


%%%% Style part, chapter, section, subsection, subsubsection heading and table of contents
% For table of contents
\contentsmargin[1cm]{0cm}
\titlecontents{chapter}[0em]{\color{steelblue}\mmpdkb}
{\thecontentslabel\enspace}
{\hspace{1.05em}}
{ \hfill\contentspage}[\vskip 6pt]

\titlecontents{section}[1em]{\color{steelblue}\mmpdk}
{\thecontentslabel\enspace}
{}
{\titlerule*[1pc]{.}\quad\contentspage}[\vskip 4pt]

\titlecontents{subsection}[2.7em]{\color{steelblue}\mmpdk}
{\thecontentslabel\enspace}
{}
{\titlerule*[1pc]{.}\quad\contentspage}[\vskip 3pt]

% For headings
\titleformat{\section}[block] 
{\color{MidnightBlue}\large \mmpdkb}{{\color{MidnightBlue}\large\thesection}}{1em}{}

\titleformat{\subsection}[block] 
{\color{MidnightBlue}\normalsize \mmpdkbkb}{{\color{MidnightBlue}\normalsize\thesubsection}}{1em}{}

\titleformat{\subsubsection}[block] 
{\color{MidnightBlue}\normalsize\mmpdkbkb }{{\color{MidnightBlue}\normalsize\thesubsubsection}}{1em}{}

\titleformat{\chapter}[display] 
    {\color{MidnightBlue}\LARGE\mmpdkb}{\color{MidnightBlue}{\Huge\mmpdkb\thechapter}}{1ex}
    {\filright} 

\titleformat{\part}[display] 
{\filcenter\LARGE\mmpdkbkb}{{\large\partname\; \thepart}}{1em}{\thispagestyle{empty}}

%%% Code listing related
\newfontface\pgcode[Scale=0.9, BoldFont={Ubuntu Mono Bold}]{Ubuntu Mono}
\lstset{
  language     = Java,
  basicstyle   = \pgcode,
  commentstyle = \color{gray},
  keywordstyle = \color{darkgray}\textbf,
  stringstyle  = \color{green!70!black},
  stringstyle  = \color{red},
  columns      = fullflexible,
  numbers      = left,
  numberstyle  = \scriptsize\sffamily\color{gray},
  caption      = A hello world program in Java,
  showstringspaces = false,
}

%%% Custom utility commands
\newcommand*\cleartoleftpage{%
  \clearpage
  \ifodd\value{page}\hbox{}\newpage\fi
}

\newcommand{\Fig}{ပုံ\xspace}
\newcommand{\enprogramming}{ပရိုဂရမ်းမင်း{\txtmyparaen{(programming)}}}
\newcommand{\mmprogramming}{ပရိုဂရမ်းမင်း}
\newcommand{\enprogram}{ပရိုဂရမ်{\txtmyparaen{(program)}}}
\newcommand{\mmprogram}{ပရိုဂရမ်}
\newcommand{\encommand}{ကွန်မန်း{\txtmyparaen{(command)}}}
\newcommand{\mmcommand}{ကွန်မန်း}
\newcommand{\encorner}{ကွန်နာ{\txtmyparaen{(corner)}}}
\newcommand{\mmcorner}{ကွန်နာ}
\newcommand{\embeeper}{ဘိပါ{\txtmyparaen{(beeper)}}}
\newcommand{\mmbeeper}{ဘိပါ}


\title{\mmpdkbkb အခြေခံ ပရိုဂရမ် ရေးနည်း } 
\author{ပြည်စိုး} 
\date{၀၅/၀၅/၂၀၂၂}

\begin{document}

\setcounter{secnumdepth}{3} % counter for susubsection
\setcounter{tocdepth}{3} % to view susubsection in toc

\frontmatter
\maketitle
\tableofcontents
\cleardoublepage

\mainmatter

\chapter{\myenchttl{Programming with Karel the Robot}}
%https://tex.stackexchange.com/questions/10555/hyperref-warning-token-not-allowed-in-a-pdf-string
\XeTeXlinebreaklocale "my_MM"  %Myanmar line and character breaks
\XeTeXinterwordspaceshaping=2
\begin{sloppypar}



\enprogramming\ ဆိုတာဘာလဲရှင်းပြဖို့ စာတွေအများကြီးရေးနေတာထက် လက်တွေ့ \enprogram လေးတွေ စရေးကြည့်ပြီးမှ \mmprogramming ဆိုတာ ဘာလဲ ပြောပြလိုက်ရင် ပိုပြီးနားလည် သဘောပေါက်လွယ်တယ်။ 

\begin{figure}[h]
    \caption{ကားရဲလ်နှင့်  ကားရဲလ်၏ကမ္ဘာ}\label{fig:meet_karel}
    \includegraphics[width=4in, left]{ch01/meet_karel.jpg}
\end{figure}

\begin{figure}[tbh!]
    \caption{\mmbeeper နေရာရွှေ့ပြီး}\label{fig:meet_karel_aft}
    \includegraphics[width=4in, left]{ch01/meet_karel_aft.jpg}
\end{figure}

\section{ကားရဲလ်နှင့် မိတ်ဆက်ခြင်း}
ဒါ့ကြောင့် စက်ရုပ်ကားရဲလ်\myen{(Karel the Robot)}ကို သူရောက်နေတဲ့ ကမ္ဘာမှာ အလုပ်တွေ လုပ်ခိုင်းဖို့ \mmprogram လေးတွေ  အရင်ဆုံး ရေးကြည့်ကြမယ်။ \Fig \vref*{fig:meet_karel} မှာ ပြထားတာက ကားရဲလ် ရောက်ရှိနေမယ့် နမူနာ ကမ္ဘာပါ။ ကားရဲလ်ရဲ့ ကမ္ဘာထဲက မီးခိုးရောင် ဆနွင်းမကင်းကွက်ပုံ အရာကို \enbeeper\ လို့ ခေါ်တယ်။ 

 ကားရဲလ်က စကားပြောပြီး ခိုင်းလို့မရဘူး။ သူနားလည်တဲ့ \encommand တွေကို \mmprogram ရေးပြီးပဲ ခိုင်းလို့ရတယ်။ 

\section{ကားရဲလ်နားလည်သော \mmcommand လေးခု}
ကားရဲလ်က အခြေခံအားဖြင့် \mmcommand လေးခုကိုပဲ နားလည်တယ်။ \mycode{move, turnLeft, putBeeper, pickBeeper} \mmcommand တို့ ဖြစ်တယ်။ \mycode{move} ကွန်မန်းက သူရပ်နေတဲ့ \mmcorner ကနေ ရှေ့တည့်တည့် ကပ်ရပ် \mmcorner ကို ရွှေ့ခိုင်းတာ။ \mmcorner ‌တွေကို အပေါင်းသင်္ကေတ လေးတွေနဲ့ ပြထားတယ်။ ပုံထဲမှာ သေးနေတဲ့အတွက် အစက်လေးတွေလို့ ထင်ရတယ်။ 

ကားရဲလ်ကို \mycode{putBeeper} ကွန်မန်းပေးလိုက်ရင်တော့ ကားရဲလ်က သူရှိနေတဲ့ ကွန်နာမှာ ဘိပါ\myen{(beeper)} လို့ခေါ်တဲ့ အတုံးလေး တစ်ခုချထားလိမ့်မယ်။ 

\mycode{pickBeeper} ကွန်မန်းက ဘိပါကောက်ခိုင်းတာပါ။ ကားရဲလ်ရောက်နေတဲ့ ကွန်နာမှာ ဘိပါရှိရင် ဒီကွန်မန်းနဲ့ ကောက်ခိုင်းလို့ရတယ်။ 

\begin{figure}[tbh!]
    \begin{subfigure}[t]{0.46\textwidth}
        \adjincludegraphics[height=1.18in,trim={0 0 {.44\width} {.55\height}}, clip, left]{ch01/bef_move.jpg}
        \caption{ကနဦး အနေအထား}
    \end{subfigure}
    \hspace{0.1in}
    \begin{subfigure}[t]{0.46\textwidth}
        \adjincludegraphics[height=1.18in,trim={0 0 {.44\width} {.55\height}}, clip, left]{ch01/aft_1st_move.jpg}
        \caption{\myttlcode{move} \mmcommand လုပ်ဆောင်ပြီး}
    \end{subfigure}

    \begin{subfigure}[t]{0.46\textwidth}
        \adjincludegraphics[height=1.18in,trim={0 0 {.44\width} {.55\height}}, clip, left]{ch01/aft_2nd_move.jpg}
        \caption{\myttlcode{move} \mmcommand တစ်ခါထပ်၍ လုပ်ဆောင်ပြီး}
    \end{subfigure}
    \hspace{0.12in}
    \begin{subfigure}[t]{0.46\textwidth}
        \adjincludegraphics[height=1.18in,trim={0 0 {.44\width} {.55\height}}, clip, left]{ch01/aft_pick.jpg}
        \caption{\myttlcode{pickBeeper} \mmcommand လုပ်ဆောင်ပြီး}
    \end{subfigure}

    \begin{subfigure}[t]{0.46\textwidth}
        \adjincludegraphics[height=1.18in,trim={0 0 {.44\width} {.55\height}}, clip, left]{ch01/aft_3rd_move.jpg}
        \caption{\myttlcode{move} \mmcommand တစ်ခါထပ်၍ လုပ်ဆောင်ပြီး}
    \end{subfigure}
    \hspace{0.1in}
    \begin{subfigure}[t]{0.46\textwidth}
        \adjincludegraphics[height=1.18in,trim={0 0 {.44\width} {.55\height}}, clip, left]{ch01/aft_put.jpg}
        \caption{\myttlcode{putBeeper} \mmcommand လုပ်ဆောင်ပြီး}        
    \end{subfigure}

    \caption{\myttlcode{move, pickBeeper, putBeeper} \mmcommand အသီးသီး အလုပ်လုပ်ပုံ}
    \label{fig:bef_and_after_commands}
\end{figure}

\mycode{turnLeft} ကွန်မန်းကတော့  ဘယ်ဘက်လှည့်ခိုင်းတာ။ မူလအနေအထားကနေ \mycode{turnLeft} ကွန်မန်းပေးလိုက်တဲ့ အခါမှာ မျက်နှာမူရာ အရပ် ပြောင်းသွားပုံကို \Fig \vref*{fig:bef_and_after_turnLeft} တွင်ကြည့်ပါ။ ညာဘက်လှည့်ခိုင်းဖို့ ကားရဲလ်ကိို \mycode{turnRight} \mmcommand နဲ့တော့ ခိုင်းလို့မရဘူး။ သူက \mycode{turnRight}  \mmcommand ကို နားမလည်ပါဘူး။

\begin{figure}[tbh!]
    \begin{subfigure}[t]{0.4\textwidth}
        \includegraphics[scale=0.2, left]{ch01/karel_icon_E.png}
        \caption{မူလ အနေအထား(အရှေ့ဘက်လှည့်)}    
    \end{subfigure}
    \begin{subfigure}[t]{0.4\textwidth}
        \includegraphics[scale=0.2, left]{ch01/karel_icon_N.png}
        \caption{\myttlcode{turnLeft} တစ်ကြိမ်လုပ်ဆောင်ပြီး}
    \end{subfigure}
    \begin{subfigure}[t]{0.4\textwidth}
        \includegraphics[scale=0.2, left]{ch01/karel_icon_W.png}
        \caption{\myttlcode{turnLeft} တစ်ကြိမ် ထပ်၍လုပ်ဆောင်ပြီး}
    \end{subfigure}
    \begin{subfigure}[t]{0.4\textwidth}
        \includegraphics[scale=0.2, left]{ch01/karel_icon_S.png}
        \caption{\myttlcode{turnLeft} တစ်ကြိမ် ထပ်၍လုပ်ဆောင်ပြီး}
    \end{subfigure}
    \begin{subfigure}[t]{0.4\textwidth}
        \includegraphics[scale=0.2, left]{ch01/karel_icon_E.png}
        \caption{\myttlcode{turnLeft} နောက်ဆုံးတစ်ကြိမ် လုပ်ဆောင်ပြီး}
    \end{subfigure}
    \caption{ကနဦး အရှေ့ဘက် မျက်နှာမူနေရာမှ}
    \label{fig:bef_and_after_turnLeft}
\end{figure}

\section{\mymmchsec{}{ပထမဆုံး ကားရဲလ်} \mmprogram}
\enbeeper ကို မူလနေရာကနေ \Fig \vref*{fig:meet_karel_aft} မှာပြထားတဲ့ နေရာကို ရွှေ့ခိုင်းရမှာပါ။ \mmcommand တွေကို ဘယ်လို အစဉ်အတိုင်းပေးရမလဲ။ \mmbeeper ရှိတဲ့နေရာကိုသွားပြီး \mmbeeper ကောင်ခိုင်းဖို့က \mycode{move, move, move, pickBeeper} \mmcommand ပေးမယ်။ ပြီးတဲ့အခါ ရှေ့မှာနံရံရှိနေတယ်။ ဆက်သွားလို့မရတော့ဘူး။ မြောက်ဘက်ကို သွားခိုင်းဖို့ \mycode{turnLeft}။ နှစ်ကြိမ်ထပ်ပြီးရွှေ့ခိုင်းရမယ်ဆိုတော့ \mycode{move, move}။ အရှေ့ဘက် ပြန်လှည့်ခိုင်းဖို့ ညာဘက်လှည့်ခိုင်းရမယ်။ ဒါပေမယ့် \mycode{turnRight} \mmcommand မရှိဘူး။ \mycode{turnLeft} သုံးခါလုပ်ခိုင်းလိုက်ရင်လည်း အရှေ့ဘက်လှည့်သွားမှာပဲ။ တစ်ခါထပ်ပြီး \mycode{move} ခိုင်းလိုက်ရင် \mmbeeper ကိုထားခိုင်းချင်တဲ့နေရာ ရောက်ပြီ။ \mycode{putBeeper} နဲ့ \mmbeeper ချထားခိုင်းလိုက်မယ်။ \mmbeeper ကိုမြင်သာအောင် \mycode{move} တစ်ခါ ထပ်လုပ်ခိုင်းလိုက်မယ်။

အစဉ်အတိုင်း ပေးရမယ့် \mmcommand တွေက \mycode{move, move, move, pickBeeper, turnLeft, move, move, turnLeft, turnLeft, turnLeft, move, putBeeper, move} ဖြစ်မယ်။ ဒီအစီအစဉ်အတိုင်း \mmcommand တွေကို \mmprogram ရေးပေးရမယ်။ \enJPL ကို အသုံးပြုပြီး ရေးရမှာပါ။ 

\mmprogram ရေးဖို့အတွက် \encomputer\ နားလည်တဲ့ ဘာသာစကား တစ်မျိုးမျိုးကို အသုံးပြုပြီး ရေးရတယ်။ \mmcomputer\ နားလည်တဲ့၊ \mmcomputer ပရိုဂရမ်ရေးဖို့ အသုံးပြုတဲ့ ဘာသာစကားကို \enPL\ လို့ ခေါ်တယ်။ 

မြန်မာ၊ အင်္ဂလိပ် စတဲ့ လူတွေရဲ့ ဘာသာစကားတွေမှာ သဒ္ဒါရှိသလိုပဲ \mmPL တွေမှာလည်း သဒ္ဒါရှိတယ်။ \mmprogram ရေးတဲ့အခါ အသုံးပြုတဲ့ \mmPL ရဲ့ ရေးပုံရေးနည်း၊ စည်းမျဉ်းစည်းကမ်း သတ်မှတ်ချက်တွေကို လိုက်နာဖို့လိုတယ်။ \mmPL ရဲ့ သဒ္ဒါကို \ensyntax\ လို့ ခေါ်တယ်။

ဒီအခန်းအတွက်က \mmsyntax အသေးစိတ် နားလည်ဖို့ မလိုသေးဘူး။ \enJPL ရဲ့ လိုအပ်ချက်အရ ပုံစံချပေးထားတဲ့ ကားရဲလ် \mmprogram\ \entemplate ကိုအသုံးပြုပြီး \mmcommand တွေကို \entemplate ထဲမှာပဲဖြည့်ရေးမယ်။ \entemplate က အခုလိုပုံစံပါ။

\begin{lstcodesimple}[float, caption=ကားရဲလ် ပရိုဂရမ် template, label={lst:Karel_template}]
public class ClassName extends stanford.karel.Karel { 
        public void run() {
                //Karel commands
        }        
}
\end{lstcodesimple}

\noindent \mycode{ClassName} နေရာမှာ ရေးမယ့် \mmprogram အမည်ကို အစားထိုးရမယ်။ \mycode{MeetKarel} လို့ ‌ပေးရအောင်။
\begin{lstcodeminimal}[]
public class MeetKarel extends stanford.karel.Karel { 
        public void run() {
                //Karel commands
        }        
}
\end{lstcodeminimal} 

\mmcommand တွေကိုတော့ \mycode{public void run()} နဲ့ဆိုင်တဲ့ \mmcurlybrpair အတွင်းမှာ ရေးပေးရမယ်။ ဖတ်ရလွယ်အောင် တစ်လိုင်းမှာ တစ်ခုပဲရေးလေ့ရှိတယ်။ \mycode{move} \mmcommand ကို \enJPL \mmsyntax နဲ့ကိုက်ညီအောင် \mycode{move()} လို့ရေးဖို့ လိုအပ်တယ်။ \mmcommand တစ်ခုအပြီးမှာ \ensemicol\ ထည့်ပေးဖို့လဲလိုတယ်။ ဝါကျတစ်ကြောင်းဆုံးရင် ပုဒ်မချပေးရတဲ့ သဘောပဲ။ \mmcommand တစ်ခု ပြီးတိုင်း \mmsemicol\ ထည့်ပေးဖို့လိုတယ်။
\begin{lstcodeminimal}[]
public class MeetKarel extends stanford.karel.Karel { 
        public void run() {
                move();
        }        
}
\end{lstcodeminimal} 

\noindent \mmbeeper ရွှေ့ဖို့အတွက် \mmprogram အပြည့်အစုံကို \Lst \vref*{lst:meet_karel} မှာ ကြည့်ပါ။ \entemplate ဖွဲ့စည့်တည်ဆောက်ထားပုံ၊ အသုံးပြုထားတဲ့ \mycode{public, class, extends, void} စတဲ့ စကားလုံးတွေရဲ့ အဓိပ္ပါယ် စတာတွေ နားလည်ဖို့ လိုနေသေးပေမယ့် လောလောဆယ်ဆယ်တော့ ဒါတွေခဏထားပြီး ကားရဲလ်ကို ခိုင်းထားတဲ့ \mmcommand တွေဟာ \mmbeeper ကို နေရာမှန်အောင် ရွှေ့ပေးဖို့အတွက် မှန်ကန်လား၊ ပြဿနာတစ်ခုခု ရှိနေမလား သိရအောင် \mmprogram ကို \mmrun ကြည့်ရအောင်။ \mmprogram က လိုင်းအရေအတွက် နည်းပေမယ့် မှားနိုင်ပါတယ်။  \mmbeeper ရှိတဲ့ \mmcorner\ ကို မရောက် သေးပဲ ကောက်ခိုင်းတာ၊ အရပ်မျက်နှာ မှားပြီး လှည့်ခိုင်းမိတာ၊ အကြိမ်အရေအတွက် လိုနေတာ၊ ပိုသွားတာ စသည်ဖြင့် မှားနိုင်တဲ့ အချက်တွေ အများကြီးပါ။ 
\begin{lstcodesimple}[float, caption=ကားရဲလ် ပရိုဂရမ် template, label={lst:meet_karel}]
public class MeetKarel extends stanford.karel.Karel { 
        public void run() {
                move();
                move();
                move();
                pickBeeper();
                turnLeft();
                move();
                move();
                turnLeft();
                turnLeft();
                turnLeft();
                move();
                putBeeper();
                move();
        }        
}
\end{lstcodesimple}



\section{IntelliJ}
\mmprogram\ ‌ရေးဖို့ \enintellij \enIDE ကို ဖွင့်။


\subsection{\myttlcode{class} သတ်မှတ်ခြင်း}
\mycode{class} တစ်ခုသတ်မှတ်ဖို့အတွက် 

\begin{lstcodesimple}[caption=\myttlcode{class} သတ်မှတ်သည့် ဆင်းတက်စ်]
    public class MeetKarel extends stanford.karel.Karel{
            
    }
\end{lstcodesimple}

\subsection{\myttlcode{class} သတ်မှတ်ခြင်း}
\mycode{class} တစ်ခုသတ်မှတ်ဖို့အတွက် 

\begin{lstcodesimple}[caption=\myttlcode{class} သတ်မှတ်သည့် ဆင်းတက်စ်]
    public class MeetKarel extends stanford.karel.Karel{
            
    }
\end{lstcodesimple}

\section{Karel's World}
\Fig \vref*{fig:meet_karel} 
အနောက်မှအရှေ့ \mmcorner တွေ တလျှောက်က \enstreet တွေ ဖြစ်ပြီး တောင်မှမြောက် \mmcorner တွေ တလျှောက်က \enavenue တွေ ဖြစ်တယ်။ 

\mmcorner တွေက \mmavenue နဲ့ \mmstreet တွေဆုံတဲ့ လမ်းဆုံတွေ ဖြစ်တယ်။ \mmcorner တစ်ခုကို ရည်ညွှန်းဖို့ \mmavenue နဲ့ \mmstreet နံပါတ်‌တွေကို အသုံးပြုမယ်။ ကားရဲလ်က (၁,၁) \mmcorner၊ \mmbeeper က (၃,၁) \mmcorner မှာရှိနေတယ်။ ပထမ‌‌နံပါတ်က \mmavenue ၊ ဒုတိယက \mmstreet နံပါတ်ပါ။ “\mmbeeper က (၃,၁) \mmcorner မှာရှိတယ်”  လို့ရေးရင် \mmbeeper က နံပါတ် ၃ \mmavenue\ နဲ့ ၁ \mmstreet ဆုံတဲ့ \mmcorner မှာ ရှိနေတယ်လို့ ဆိုလိုတာ။



\section{ကားရဲလ် ပရိုဂရမ်ကို ခွဲခြမ်းစိတ်ဖြာကြည့်ခြင်း}

\begin{lstcodesimple}[language = Java, float=ht]
    class ClassName {
            
    }
\end{lstcodesimple}

\begin{lstcodesimple}[language = Java, float]
    public class MeetKarel{
            
    }
\end{lstcodesimple}

\begin{lstcodesimple}[language = Java, float=hb]
    public class MeetKarel extends stanford.karel.Karel{
            
    }
\end{lstcodesimple}

\begin{lstcodesimple}[language = Java, float]
    import stanford.karel.Karel;
    public class MeetKarel extends {
            
    }
\end{lstcodesimple}

\begin{lstcodesimple}[language = Java, float]
    public class MeetKarel extends stanford.karel.Karel{
            public void run(){

            }            
    }
\end{lstcodesimple}

\end{sloppypar}

\chapter{\myenchsec{Control Flow} \mymmchsec{ကွန်ထရိုးလ်ဖလိုး}}
\XeTeXlinebreaklocale "my_MM"  %Myanmar line and character breaks
\XeTeXinterwordspaceshaping=2
\begin{sloppypar}
ကားရဲလ်ကို ရှေ့ကို ၂၅ လှမ်း ရွှေ့ခိုင်းချင်တယ်။ \mycode{move(); move(); move(); ...move();} ၂၅ ခါ ရေးလို့‌တော့ရတာပေါ့။ ဒါပေမယ့် စာရိုက်ရတာ အချိန်လည်းကုန် လက်လဲညောင်း ဖြစ်မယ်။ {\mycode{move();}} ကို ၂၅ ကြိမ် ကျော့ပေးပါလို့ ခိုင်းလို့ရရင် ပိုမကောင်းဘူးလား။ 

ကားရဲလ်ရဲ့ ရှေ့တည့်တည့် ခပ်လှမ်းလှမ်းမှာ နံရံတစ်ခုရှိနေမယ်။ ဘယ်လောက်ဝေးလဲ မသိဘူးဆိုပါစို့။ ကားရဲလ်ကို နံရံဆီရောက်အောင် သွားခိုင်းချင်တယ်။ နံရံက ဘယ်လောက်ဝေးလဲ မသိတော့ {\mycode{move}} ကို ဘယ်နှစ်ကြိမ် ကျော့ခိုင်းရမလဲ မသိနိုင်ဘူး။ ရှေ့မှာရှင်းနေသေးသ၍ \mycode{move} ပါလို့သာ ခိုင်းလို့ရမယ်ဆိုရင် တော်တော်လေး အဆင်ပြေပြီ။  

အခြေအနေတစ်ခု မှန်တော့မှပဲ \mmcommand တွေကို  လုပ်ဆောင်စေချင်တာမျိုးလဲ ရှိတယ်။ အဲဒီ အခြေနေနဲ့ မကိုက်ညီဘူး (တနည်းအားဖြင့် အခြေအနေက မှားနေခဲ့ရင်) \mmcommand တွေကို မလုပ်ဆောင်ပဲ ကျော်သွားစေချင်တယ်။ ရှေ့မှာပြောခဲ့တဲ့ နံရံအောက််ခြေမှာ \mmbeeper တစ်ခု ရှိနေနိုင်တယ်၊ ရှိချင်မှလည်း ရှိမယ်ဆိုပါစို့။ \mmbeeper ရှိနေခဲ့ရင် နံရံ အခြားတဘက်မှာ \mmbeeper ကို ထားခိုင်းချင်တယ်။  ဒါဆိုရင် \mmbeeper ရှိနေမှပဲ အခြားတဘက်ကို ရွှေ့ခိုင်းဖို့ လိုအပ်တဲ့ \mmcommand တွေကို လုပ်ဆောင်စေချင်တယ်။ \mmbeeper မရှိဘူးဆိုရင် အဲဒီ \mmcommand တွေကို မလုပ်ဆောင်စေချင်ဘူး။  

နောက်ထပ်တစ်မျိုးက အခြေအနေတစ်ခု မှန်ခဲ့ရင် လုပ်ဆောင်စေချင်တဲ့ \mmcommand တွေနဲ့ မှားခဲ့ရင်‌ လုပ်ဆောင်စေချင်တဲ့ \mmcommand ‌တွေကို မတူပဲဖြစ်နေတာမျိုးပါ။ တနည်းအားဖြင့် အခြေအနေပေါ် မူတည်ပြီး ခိုင်းရမယ့် အလုပ်ကမတူဘူး။ \mmbeeper ရှိခဲ့ရင် နံရံရဲ့ အခြားဘက်ကိုရွှေ့ခိုင်းချင်တယ်၊ မရှိခဲ့ရင်တော့ လာလမ်းအတိုင်း ပြန်လာစေချင်တယ် ဆိုပါစို့။ ဒါဆိုရင် \mmbeeper ရှိခြင်း၊ မရှိခြင်းပေါ် မူတည်ပြီး လုပ်ဆောင်ရမယ့် \mmcommand တွေက  မတူဘူး။ 

ဒီအခန်းမှာတော့ အထက်ပါ လိုအပ်ချက်မျိုးတွေအတွက် အသုံးပြုတဲ့ \enControlFlowStatements တွေကို လေ့လာကြမယ်။ ပြန်ကျော့ဖို့အတွက် သုံးတဲ့ \mycode{for} \myen{loop} နဲ့ \mycode{while} \myen{loop} ကို အရင်ကြည့်ကြရအောင်။

\section{{\mycodesecttl{for}} \myensecttl{loop}}
\mycode{for} \myen{loop} ကို \mmcommand တစ်ခု သို့ တစ်စုကို ပြန်ကျော့ဖို့ အသုံးပြုနိုင်တယ်။ \mmsyntax က အခုလိုပုံစံမျိုး နဲ့ရေးရတယ်  
\begin{lstcodeminimal}
for (int i = 0; i < N; i++) {
    //one or more commands here
}
\end{lstcodeminimal}
\mycode{move} ကို နှစ်ဆယ့်ငါးကြိမ် ကျော့ချင်ရင် 
\begin{lstcodeminimal}[]
for (int i = 0; i < 25; i++) {
    move();
}
\end{lstcodeminimal}
ပြန်ကျော့ချင်တဲ့ \mmcommand တွေကို \mmcurlybrpair အတွင်း လိုင်းတွေမှာရေးပြီး \mycode{N} နေရာမှာ အကြိမ်အရေအတွက်ကို အစားထိုးပေးရမယ်။ \mycode{int i} မှာ ခြားထားရပါမယ်။ \mycode{inti} ဆိုရင် \mmsyntaxerr ဖြစ်မယ်။ \mycode{i++} က တဆက်ထဲ ဖြစ်ရမယ်။ \mycode {i ++} သို့ \mycode {i + +} သို့ \mycode {i+ +} ရေးလို့ မရဘူး။ ကျန်တဲ့ စပေ့စ်တွေက ခြားသည်ဖြစ်စေ မခြားသည်ဖြစ်စေ ပြဿနာ \mmsyntax\ အရတော့ မှန်တယ်။ ဒီလိုရေးလို့ ရပေမယ့် ဖတ်ရတာ ခက်တယ်။ 
\begin{lstcodeminimal}[]
for (int i=0;i<25;i++){
move();
}
\end{lstcodeminimal}
\noindent ပူးကပ်နေပြီး အမြင်အရလည်း မရှင်းလင်းဘူး။

\subsection{\mycodesubsecttl{for} \myensubsecttl{loop} အသုံးပြုသည့် ဥပမာများ}
\begin{figure}[htb]
    \caption{ကားရဲလ်နှင့်  ကားရဲလ်၏ကမ္ဘာ}\label{fig:MakeRowOfFiveBeepersInit}
    \includegraphics[scale=0.2, left]{ch02/MakeRowOfFiveBeepers/init.jpg}
\end{figure}
ကားရဲလ်က \Fig \vref*{fig:MakeRowOfFiveBeepersInit} ကမ္ဘာထဲမှာရှိနေမယ်။ လမ်းတလျှောက်လုံး \mmcorner တစ်ခု စီတိုင်းမှာ \mmbeeper တစ်ခု ချထားခိုင်းချင်တယ်။ \mmavenue အရေအတွက်က ဒီပုစ္ဆာအတွက် အပြောင်းအလဲမရှိဘူး။ ပုံသေ ၅ ခုပဲဖြစ်မယ်လို့ ယူဆပါ။ \mmbeeper ချလိုက် ရှေ့\mmcorner ကို ရွှေ့လိုက်၊ \mmbeeper ချလိုက် ရှေ့\mmcorner ကို ရွှေ့လိုက် လုပ်ခိုင်းရမှာပေါ့။  ပြန်ကျော့ပေးရမယ့် \mmcommand နှစ်ခုက \mycode{putBeeper} နဲ့ \mycode{move}။ ဘယ်နှစ်ကြိမ် ကျော့ခိုင်းရမှာလဲ။ သေချာဖို့လိုတယ်။ သေချာတာက ကားရဲလ်ရှေ့မှာ \mmcorner လေးခုပဲရှိတဲ့အတွက် ရှေ့ကို လေးနေရာပဲ ရွှေ့လို့ရမယ်။ \mycode{N} နေရာမှာ \mycode{4} ထည့်ပြီး အခုလို ရေးရမယ်။ \Lst \vref*{lst:MakeRowOfFiveBeepersOffByOne} တွင်ကြည့်ပါ။

 % Ex -> start with putBeeper
\begin{lstcodesimple}[float, caption=ပထမစမ်းကြည့်ပုံ, label={lst:MakeRowOfFiveBeepersOffByOne}]
public class MakeRowOfFiveBeepers extends stanford.karel.Karel{
    public void run(){
            for(int i = 0; i < 4; i++) {
                    putBeeper();
                    move();
            }
    }
}
\end{lstcodesimple}

ဒီတိုင်း \mmrun ကြည့်တဲ့ရအောင်။ ရလဒ်ကို \Fig \vref{fig:MakeRowOfFiveBeepersOffByOne} တွင်ကြည့်ပါ။  နောက်ဆုံး \mmcorner မှာ \mmbeeper ချဖို့ကျန်နေတာ တွေ့ရမယ်။ ဘာကြောင့်ကျန်နေရတာလဲ။ သေချာ နားလည်အောင် \mmiteration တစ်ခါပြီးတိုင်း ရှိနေမယ့် အခြေအနေကို \vref*{fig:MakeRowOfFiveBeepersIters} မှာကြည့်ပါ။ နောက်ဆုံး \mmcorner မှာ \mmbeeper ချပေးဖို့အတွက် \mycode{for} \myen{loop} \mmClosingCurlyBrace\ နောက်တစ်လိုင်းမှာ \mycode{putBeeper();} \mmcommand ရေးပေးရမယ်။ \enForLoopBody အပြင်မှာရှိတဲ့အတွက် ပြန်ကျော့မယ့် \mmcommand\ ထဲမှာ မပါဘူး။ နောက်ဆုံးမှာ တစ်ကြိမ်ပဲ လုပ်ဆောင်တယ်။ \Lst \vref*{lst:MakeRowOfFiveBeepersFixedOffByOne} တွင်ကြည့်ပါ။

\begin{figure}[htb]
    \caption{ကားရဲလ်နှင့်  ကားရဲလ်၏ကမ္ဘာ}\label{fig:MakeRowOfFiveBeepersOffByOne}
    \includegraphics[scale=0.2, left]{ch02/MakeRowOfFiveBeepers/off_by_one_1.jpg}
\end{figure}

\begin{figure}[tbh!]
    \begin{subfigure}[t]{0.8\textwidth}
        \includegraphics[scale=0.2, left]{ch02/MakeRowOfFiveBeepers/init.jpg}
        \caption{မူလ အနေအထား}    
    \end{subfigure}
    \begin{subfigure}[t]{0.8\textwidth}
        \includegraphics[scale=0.2, left]{ch02/MakeRowOfFiveBeepers/1st_iter.jpg}
        \caption{ပထမ တစ်ကျော့ပြီး}    
    \end{subfigure}
    \begin{subfigure}[t]{0.8\textwidth}
        \includegraphics[scale=0.2, left]{ch02/MakeRowOfFiveBeepers/2nd_iter.jpg}
        \caption{ဒုတိယ တစ်ကျော့ပြီး}    
    \end{subfigure}
    \begin{subfigure}[t]{0.8\textwidth}
        \includegraphics[scale=0.2, left]{ch02/MakeRowOfFiveBeepers/3rd_iter.jpg}
        \caption{တတိယ တစ်ကျော့ပြီး}    
    \end{subfigure}
    \begin{subfigure}[t]{0.8\textwidth}
        \includegraphics[scale=0.2, left]{ch02/MakeRowOfFiveBeepers/4th_iter.jpg}
        \caption{စတုတ္ထမြောက် ကျော့ပြီး}    
    \end{subfigure}
    \caption{\mmiteration တစ်ခုပြီး}
    \label{fig:MakeRowOfFiveBeepersIters}
\end{figure}

\begin{lstcodesimple}[float, caption={ပထမစမ်းကြည့်ပုံ}, 
                        label={lst:MakeRowOfFiveBeepersFixedOffByOne}]
public class MakeRowOfFiveBeepers extends stanford.karel.Karel{
    public void run(){
            for(int i = 0; i < 4; i++) {
                    putBeeper();
                    move();
            }

            putBeeper();
    }
}
\end{lstcodesimple}

\Fig \vref{subfig:MakeBeeperSquareInit} မှ စတုရန်းပုံ ကမ္ဘာထဲမှာ \mmcorner တွေမှာ \mmbeeper တစ်ခုစီ ချဖို့ \enForLoop ကိုသုံးနိုင်တယ်။  \Lst \vref*{lst:MakeBeeperSquare} တွင်ကြည့်ပါ။ \mmiteration တစ်ခါတိုင်းမှာ 
\begin{lstcodeminimal}[]
putBeeper();
move();
turnLeft();    
\end{lstcodeminimal}

\noindent ကို လုပ်ဆောင်မှာဖြစ်တယ်။ တစ်ကြိမ်ကျော့အပြီးမှာ ရှိနေမယ့် အခြေအနေကို \Fig \vref*{fig:MakeBeeperSquareIters} တွင်ကြည့်ပါ။
\begin{lstcodesimple}[float, caption={စတုရန်းပုံ ဘိပါ}, label={lst:MakeBeeperSquare}]
public class MakeBeeperSquare extends stanford.karel.Karel{
        public void run(){
                for(int i = 0; i < 4; i++) {
                        putBeeper();
                        move();
                        turnLeft();
                }
        }
}
\end{lstcodesimple}

\begin{figure}[tbh!]
    \caption{\myenlstcpt{Beeper Seqare}}
    \begin{subfigure}[t]{0.46\textwidth}
        \includegraphics[scale=0.17, left]{ch02/MakeBeeperSquare/init.jpg}
        \caption{ကနဦး အနေအထား}
        \label{subfig:MakeBeeperSquareInit}
    \end{subfigure}
    \hspace{0.1in}
    \begin{subfigure}[t]{0.46\textwidth}
        \includegraphics[scale=0.17, left]{ch02/MakeBeeperSquare/1st_iter.jpg}
        \caption{\myttlcode{move} \mmcommand လုပ်ဆောင်ပြီး}
    \end{subfigure}

    \begin{subfigure}[t]{0.46\textwidth}
        \includegraphics[scale=0.17, left]{ch02/MakeBeeperSquare/2nd_iter.jpg}
        \caption{ကနဦး အနေအထား}
    \end{subfigure}
    \hspace{0.1in}
    \begin{subfigure}[t]{0.46\textwidth}
        \includegraphics[scale=0.17, left]{ch02/MakeBeeperSquare/3rd_iter.jpg}
        \caption{\myttlcode{move} \mmcommand လုပ်ဆောင်ပြီး}
    \end{subfigure}

    \begin{subfigure}[t]{0.46\textwidth}
        \includegraphics[scale=0.17, left]{ch02/MakeBeeperSquare/3rd_iter.jpg}
        \caption{\myttlcode{move} \mmcommand လုပ်ဆောင်ပြီး}
    \end{subfigure}
    \hspace{0.1in}
    \label{fig:MakeBeeperSquareIters}
\end{figure}



\section{\mycodesecttl{while} \myensecttl{loop}}
အကြိမ်အရေအတွက် မသိပဲ အခြေအနေတစ်ခု မှန်နေသ၍ ပြန်ကျော့ဖို့ \enWhileLoop ကို သုံးတယ်။ \mmsyntax က အောက်ပါအတိုင်းရေးတယ်။  
\begin{lstcodeminimal}
while (CONDITION) {
    //commands to be executed while condition is true
}
\end{lstcodeminimal}
\myen{condition} နေရာမှာ \vref*{tbl:KarelConditions} ပါ \mmCondition တစ်ခုကို အစားထိုးပေရမှာပါ။ ရှေ့မှာ ရှင်းနေသ၍ ရွှေ့ခိုင်းချင်တယ် ဆိုရင်
\begin{lstcodeminimal}
while (frontIsClear()) {
    move();
}
\end{lstcodeminimal}

\subsection{\mycodesecttl{while} \myensecttl{loop} အသုံးပြုသည့် ဥပမာများ}
ကားရဲလ် အရှေ့ဘက်မှာ အုန်းပင်တစ်ပင်ရှိတယ်။ ဘယ်လောက်အကွာအဝေးမှာ ရှိနေမလဲ ပုံသေပြောလို့မရဘူး။ နမူနာ နှစ်ခုကို ပုံမှာပြထားတယ်။ က မှာရှိနေသည်ဖြစ်စေ၊ ခ မှာရှိနေသည်ဖြစ်စေ၊ နံရံဆီကို သွားခိုင်းရမယ်။ ဒါ့အပြင် နောက်ထပ် အလားတူတဲ့  အခြားကမ္ဘာတွေထဲမှာလည်း နံရံက ဘယ်လောက်အကွာအဝေးမှာ ရှိနေသည်ဖြစ်စေ ကားရဲလ်ကို ရောက်အောင် သွားခိုင်းချင်တယ်။ 

\begin{figure}[tbh!]
    \caption{\myenlstcpt{CococonutTree}}
    \begin{subfigure}[t]{0.46\textwidth}
        \includegraphics[height=2.2in, left]{ch02/CoconutTree/a.jpg}
        \caption{}
        \label{subfig:CoCoconutTreeA}
    \end{subfigure}

    \begin{subfigure}[t]{0.46\textwidth}
        \includegraphics[height=2.2in, left]{ch02/CoconutTree/b.jpg}
        \caption{}
    \end{subfigure}
\end{figure}

\begin{lstcodesimple}[float, caption={\mycodelstcpt{CococonutTree.java} \myenlstcpt{A}}, label={lst:CococonutTree}]
public class CococonutTree extends stanford.karel.Karel{
    public void run(){
            while(frontIsClear()){
                    move();
            }
    }
}
\end{lstcodesimple}

နောက်ထပ် ဥပမာတစ်ခု ကြည့်ရအောင်။ ကားရဲလ်ကို လမ်းတလျှောက် \mmbeeper  တွေချထားခိုင်းချင်တယ်။ လမ်းအရှည်က ဘယ်လောက်ဖြစ်ဖြစ် အပြည့်ချထားပေးရမယ်။ လမ်းအရှည်က ပုံသေမဟုတ်တဲ့အတွက် ရှေ့မှာရှင်းနေသ၍ \mmbeeper ချလိုက် ရှေ့တိုးလိုက် ထပ်ခါထပ်ခါ လုပ်ခိုင်းရမှာပေါ့။ 

\begin{lstcodesimple}[float, caption={\mycodelstcpt{MakeBeeperRow.java} \myenlstcpt{A}}, label={lst:MakeBeeperRow}]
public class MakeBeeperRow extends stanford.karel.Karel{
    public void run(){
            while(frontIsClear()){
                    putBeeper();
                    move();
            }
            putBeeper();
    }
}
\end{lstcodesimple}

\enForLoop မှာလိုပဲ \myen{loop} \mmBody ကို နောက်ဆုံးအကြိမ် လုပ်ဆောင်အပြီးမှာ \mmbeeper ချဖို့ကျန်နေမယ်။ ဒါကြောင့် \enWhileLoopBody နောက်တစ်လိုင်းမှာ \mycode{putBeeper();} ပါရမှာဖြစ်တယ်။

\begin{figure}[tbh!]
    \caption{\myenlstcpt{MakeBeeperRow}}
    \begin{subfigure}[t]{0.46\textwidth}
        \includegraphics[width=2.75in, left]{ch02/MakeBeeperRow/init.jpg}
        \caption{}
        \label{subfig:MakeBeeperRowA}
    \end{subfigure}

    \begin{subfigure}[t]{0.46\textwidth}
        \includegraphics[width=2.75in, left]{ch02/MakeBeeperRow/1st_iter.jpg}
        \caption{}
    \end{subfigure}

    \begin{subfigure}[t]{0.46\textwidth}
        \includegraphics[width=2.75in, left]{ch02/MakeBeeperRow/2nd_iter.jpg}
        \caption{}
    \end{subfigure}

    \begin{subfigure}[t]{0.46\textwidth}
        \includegraphics[width=2.75in, left]{ch02/MakeBeeperRow/3rd_iter.jpg}
        \caption{}
    \end{subfigure}

    \begin{subfigure}[t]{0.46\textwidth}
        \includegraphics[width=2.75in, left]{ch02/MakeBeeperRow/4th_iter.jpg}
        \caption{}
    \end{subfigure}
    \label{fig:MakeBeeperRowIters}
\end{figure}

\enWhileLoop အလုပ်လုပ်ပုံက အခုလိုပါ။  \mmCondition ကို အရင်ဆုံး စစ်ပါတယ်။ မှန်လျှင် \mmBody ထဲက \mmcommand တွေကို လုပ်ဆောင်တယ်။ တစ်ကြိမ်ကျော့ပြီးတိုင်း \mmCondition ကိုပြန်စစ်ပါတယ်။ မှန်နေသေးလျှင် \mmBody ထဲက \mmcommand တွေကို ထပ်ပြီးကျော့ပေးပါတယ်။ ဒီလိုမျိုး \mmCondition စစ်လိုက်၊ မှန်နေရင် ပြန်ကျော့လိုက်ကို ထပ်ခါထပ်ခါ လုပ်နေပြီး \mmCondition စစ်လိုက်လို့ မှားနေတဲ့ အခါကျတော့မှ ပြန်ကျော့တာကို ရပ်လိုက်တာဖြစ်တယ်။ ပြန်ကျော့တာရပ်လိုက်တာကို \enLoopExits တယ်လို့ ပြောလေ့ရှိတယ်။

\mmLoopExits သွားပြီးနောက် \mmLoopBody ‌အောက်ကမှာရှိတဲ့လိုင်းတွေကို ဆက်ပြီးလုပ်ဆောင်မှာ ဖြစ်တယ်။ ရှေ့ကဥပမာမှာ လေးကြိမ်မြောက်ထိ \mycode{frontIsClear} \mmCondition က မှန်နေတယ်။ ဒါကြောင့် လေးခါကျော့ ခံရမယ်။ ငါးကြိမ်မြောက် \mmCondition စစ်တဲ့အခါမှာတော့ \mycode{frontIsClear} ကမှားနေပြီ။ ရှေ့မှာ နံရံပိတ်နေတယ်။ ထပ်မကျော့တော့ပဲ \mmLoopExits ပြီး နောက်တလိုင်းက \mycode{putBeeper();} ကို ဆက်လက်လုပ်ဆောင်တယ်။

\enWhileLoop \mmCondition က ဘယ်တော့မှ မမှားတော့ပဲ အမြဲမှန်နေတာမျိုးရှိနိုင်တယ်။ ဒီအခါမှာတော့ \enLoop ထဲကနေ ဘယ်တော့မှ ထွက်မသွားတော့ပဲ \mmBody ထဲက \mmcommand တွေကို ထာဝရ လုပ်ဆောင်နေတော့မှာ ဖြစ်တယ်။ ဒီလို အစဉ်အမြဲ ပြန်ကျော့နေမယ့် \mmLoop ကို \mmInfiniteLoop လို့ခေါ်ပါတယ်။ \enInfiniteLoop ထဲကနေ မထွက်တော့တဲ့ အတွက် \mmLoopBody အောက် လိုင်းတွေမှာရှိတဲ့ \mmcommand တွေကိုလည်း လုပ်ဆောင်မှာ မဟုတ်ပါဘူး။ 

\noindent \Lst \vref*{lst:GoAroundForever} မှာ \mycode{pickBeeper();} ကို ဘယ်တော့မှ လုပ်ဆောင်ဖြစ်မှာ မဟုတ်ပါဘူး။

\begin{lstcodesimple}[float, caption={\mycodelstcpt{GoAroundForever.java} \myenlstcpt{A}}, label={lst:GoAroundForever}]
public class GoAroundForever extends stanford.karel.Karel{
    public void run(){
            //Karel will never get out of this loop
            while(frontIsClear()) {
                    move();
                    turnLeft();
            }
            //This command will never be executed
            pickBeeper();
    }
}
\end{lstcodesimple}

\enWhileLoop \enCondition က ပထမဆုံး စစစ်လိုက်တဲ့ အခါမှာပဲ မှားနေရင် \mmcommand တွေကို တစ်ခါမှ မကျော့ပေးတော့ပဲ \mmLoopExits မှာဖြစ်တယ်။ ဒါကိုတော့ \enLoopNeverEntered လို့ပြောလေ့ရှိတယ်။

\begin{lstcodesimple}[float, caption={\mycodelstcpt{LoopNeverEntered.java} \myenlstcpt{A}}, label={lst:LoopNeverEntered}]
public class LoopNeverEntered extends stanford.karel.Karel{
    public void run(){
            // This loop is never entered for default world
            // and its body will be totally skipped
            while(frontIsClear()) {
                    move();
                    turnLeft();
                    putBeeper();
            }
            // Commands below will be executed as usual
            turnLeft();
            move();
            putBeeper();
    }
}
\end{lstcodesimple}

\begin{figure}[tbh!]
    \caption{\myenlstcpt{LoopNeverEntered}}
    \begin{subfigure}[t]{0.46\textwidth}
        \includegraphics[width=2.5in, left]{ch02/LoopNeverEntered/init.jpg}
        \caption{}
        \label{subfig:LoopNeverEnteredA}
    \end{subfigure}
    \hspace{0.1in}
    \begin{subfigure}[t]{0.46\textwidth}
        \includegraphics[width=2.5in, left]{ch02/LoopNeverEntered/final.jpg}
        \caption{}
    \end{subfigure}
    \label{fig:LoopNeverEntered}
\end{figure}

%|@{}p{4cm}@{}|@{}p{5cm}@{}|
\begin{table}
\begin{tabular}{@{}p{0.35\textwidth}@{}@{}p{0.35\textwidth}@{}}
\toprule[1.5pt]
\multicolumn{2}{c}{\mytblhdr{\myentblhdr{Karel Conditions}}}\\
\midrule
\mycode{frontIsClear()	} &	    \mycode{frontIsBlocked()   }\\
\mycode{leftIsClear()	} &		\mycode{leftIsBlocked()    }\\
\mycode{rightIsClear()	} &	    \mycode{rightIsBlocked()   }\\
\mycode{beepersPresent()} &		\mycode{noBeepersPresent() }\\
\mycode{beepersInBag()	} &	    \mycode{noBeepersInBag()   }\\
\mycode{facingNorth() 	} &	    \mycode{notFacingNorth()   }\\
\mycode{facingEast() 	} &		\mycode{notFacingEast()    }\\
\mycode{facingSouth() 	} &	    \mycode{notFacingSouth()   }\\
\mycode{facingWest() 	} &		\mycode{notFacingWest()    }\\
\bottomrule[1.5pt]
\caption{ကားရဲလ်နားလည်သော \mmCondition များ}
\label{tbl:KarelConditions}
\end{tabular}
\end{table}

\section{\mycodesecttl{if}  \myensecttl{statement}}
\enIfStatement ကို \mmcommand တွေကို အခြေအနေမှန်တော့မှ လုပ်ဆောင်စေချင်တဲ့ အခါသုံးတယ်။ \mmsyntax ကတော့ အခုလိုပုံစံ။
\begin{lstcodeminimal}
if ( (*@\myLstPlaceHolder{CONDITION}@*) ) {
        // commands to be executed if (*@\myLstPlaceHolder{CONDITION}@*) is true
}
\end{lstcodeminimal}
\noindent \myLstPlaceHolder{CONDITION} နေရာမှာ \Tbl \vref*{tbl:KarelConditions} ထဲက \mmCondition တစ်ခုကို အစားထိုးရမှာပါ။ လက်တွေ့ ဥပမာတစ်ခုကြည့်ရအောင်။ ကားရဲလ်က 

\begin{figure}[tbh!]
    \caption{\myenlstcpt{MooveMoveBeeperToOtherSide}}
    \begin{subfigure}[t]{0.46\textwidth}
        \includegraphics[width=2.5in, left]{ch02/MoveBeeperToOtherSide/init_w1.jpg}
        \caption{}
        \label{subfig:MooveMoveBeeperToOtherSideInitW1}
    \end{subfigure}
    \hspace{0.1in}
    \begin{subfigure}[t]{0.46\textwidth}
        \includegraphics[width=2.5in, left]{ch02/MoveBeeperToOtherSide/final_w1.jpg}
        \caption{}
    \end{subfigure}

    \begin{subfigure}[t]{0.46\textwidth}
        \includegraphics[width=2.5in, left]{ch02/MoveBeeperToOtherSide/init_w2.jpg}
        \caption{}
        \label{subfig:MooveMoveBeeperToOtherSideInitW2}
    \end{subfigure}
    \hspace{0.1in}
    \begin{subfigure}[t]{0.46\textwidth}
        \includegraphics[width=2.5in, left]{ch02/MoveBeeperToOtherSide/final_w2.jpg}
        \caption{}
    \end{subfigure}
    \label{fig:MooveMoveBeeperToOtherSide}
\end{figure}

\begin{lstcodesimple}[float, caption={\mycodelstcpt{MoveBeeperToOtherSide.java}}, label={lst:MoveBeeperToOtherSide}]
public class MoveBeeperToOtherSide extends stanford.karel.Karel {
    public void run() {

            // Code to go pick beeper

            // Check if beeper is there and move to the other side
            if (beepersPresent()) {
                    pickBeeper();
                    turnLeft();
                    move();
                    turnRight();
                    move();
                    turnRight();
                    move();
                    putBeeper();
                    turnLeft();
            }
    }

    // turnRight method definition
}

\end{lstcodesimple}

\section{\mycodesecttl{if else}  \myensecttl{statement}}
\enIfElseStatement ကိုတော့ အခြေအနေတစ်ရပ် မှန်လျှင် လုပ်ဆောင်ချင်တာနဲ့ မှားလျှင် လုပ်ဆောင်ချင်တာ က မတူပဲကွဲပြား နေတဲ့အခါ သုံးတယ်။ \mmsyntax ကတော့ အခုလိုပုံစံ။
\begin{lstcodeminimal}
if ( (*@\myLstPlaceHolder{CONDITION}@*) ) {
        // commands to be executed if (*@\myLstPlaceHolder{CONDITION}@*) is true
} else {
        // commands to be executed if (*@\myLstPlaceHolder{CONDITION}@*) is false
}
\end{lstcodeminimal}
\noindent \myLstPlaceHolder{CONDITION} နေရာမှာ \Tbl \vref*{tbl:KarelConditions} ထဲက \mmCondition တစ်ခုကို အစားထိုးရမှာပါ။ ရှေ့မှာတွေ့ခဲ့တဲ့ ဥပမာမှာလိုပဲ \mmbeeper ရှိရင် ရွှေ့ပေးရမယ်၊ မရှိရင်တော့ စထွက်တဲ့နေရာကို ပြန်လာခိုင်းရမယ်ဆိုပါစို့။ ဒါဆိုရင် ဘိပါရှိရင် ရွှေ့ရမယ်၊ မရှိခဲ့ရင် (တနည်းအားဖြင့် beepersPresent \mmCondition ကမှားခဲ့လျှင်) ပြန်လာရမှာဖြစ်တယ်။ 

\begin{lstcodesimple}[float, caption={\mycodelstcpt{MoveBeeperToOtherSideOrComeBack.java}}, label={lst:MoveBeeperToOtherSideOrComeBack}]
public class MoveBeeperToOtherSideOrComeBack extends stanford.karel.Karel {
        public void run() {
                
                // ... 

                if (beepersPresent()) {
                        pickBeeper();
                        turnLeft();
                        move();
                        turnRight();
                        move();
                        turnRight();
                        move();
                        putBeeper();
                        turnLeft();
                } else {
                        turnLeft();
                        turnLeft();
                        move();
                        move();
                        move();
                }
        }

        // . . .turnRight method definition here
        
}
\end{lstcodesimple}

\end{sloppypar}
\chapter{\myenchttl{Program Design - Solving More Complex Programs}}
\XeTeXlinebreaklocale "my_MM"  %Myanmar line and character breaks
\XeTeXinterwordspaceshaping=2
\begin{sloppypar}
ကားရဲလ်ကို ခိုင်းချင်တဲ့ အလုပ်တွေက ရှုပ်ထွေးလာတာနဲ့ အမျှ \mmprogram ရေးရတာ ပိုပြီးခက်ခဲ လာတယ်။ ဘယ်လောက်ပဲ ရှုပ်ထွေးခက်ခဲတဲ့ အလုပ်ပဲဖြစ်ပါစေ၊ တစ်ဆင့်ချင်း တစ်ပိုင်းချင်း ခွဲခြားကြည့်မယ်ဆိုရင် ရိုးရှင်းတဲ့ အလုပ်တွေနဲ့ ဖွဲစည်းထာတာပါပဲ။ ဒါကြောင့် ခက်ခဲတဲ့အလုပ်ကို ပိုပြီး ရိုးရှင်းတဲ့ အလုပ်တွေဖြစ်အောင် ခွဲထုတ်ပြီး၊ တစ်ပိုင်းချင်းစီ ဖြေရှင်းသွားမယ်ဆိုရင် ပိုပြီးလွယ်ကူတယ်။ \mmprogram ရေးပြီး ပြဿနာတွေကို ဖြေရှင်းတဲ့ အခါမှာ ဒီသဘောတရားကို အသုံးချပုံနဲ့ လုပ်ဆောင်ရမယ့် လုပ်ငန်းစဉ်တွေကို လေ့လာကြရအောင်။

\section{\myensecttl{Problem Decomposition}}
\begin{figure}[!htb]
  \caption{\mymmfigcpt{သတင်းစာ ကောက်မည့် ကားရဲလ်၏ ကမ္ဘာ}}\label{fig:PickNewspaper}
  \includegraphics[width=3in, left]{ch03/PickNewspaper/init.jpg}
\end{figure}
ကားရဲလ်ကို ခြံဝင်းတံခါးဝနားရှိ သတင်းစာကို သွားပြီး ကောက်ခိုင်းမယ်။ \Fig \vref{fig:PickNewspaper} တွင် ကြည့်ပါ။ \mmcommand တွေကို ဘယ်လို အစီအစဉ်နဲ့ ပေးရမှာလဲ မစဉ်းစားသေးပဲ ဘာတွေလုပ်ခိုင်းရမှာလဲ စဉ်းစားကြည့်လျှင်
 \begin{enumerate}
   \item သတင်းစာရှိတဲ့ နေရာကိုသွား \label{itm:PickNsprGo}
   \item သတင်းစာကောက် \label{itm:PickNsprPick}
   \item နကိုမူလနေရာကို ပြန်လာ \label{itm:PickNsprBack}
 \end{enumerate}
 စသည့် အလုပ်သုံးခု အဓိကပါဝင်တယ်လို့ ပထမ အဆင့်အနေနဲ့ အကြမ်းဖျဉ်း ခွဲခြားကြည့်လို့ရမယ်။ “\ref*{itm:PickNsprGo} သတင်းစာရှိတဲ့ နေရာကိုသွား” ဖို့အတွက် လုပ်ရမယ့် အလုပ်တွေကို နောက်ထပ်တစ်ဆင့် ထပ်၍ ခွဲခြား ကြည့်ပါက
\begin{itemize}
  \item ရှေ့တည့်တည့် နံရံဆီကိုသွား
  \item ညာဘက်လှည့်
  \item ရှေ့တိုး 
  \item ဘယ်ဘက်လှည့်
  \item ‌ရှေ့တိုး 
\end{itemize}
“\ref*{itm:PickNsprPick} သတင်းစာကောက်” ဖို့အတွက် အလုပ်တွေကို ထပ်ခွဲထုတ်ကြည့်ရင်
\begin{itemize}
  \item သတင်းစာရှိ/မရှိ စစ်
  \item ရှိလျှင်ကောက် 
\end{itemize}
“\ref*{itm:PickNsprBack} နကိုမူလနေရာကို ပြန်လာ” ဖို့အတွက် ပါဝင်မယ့် အလုပ်တွေကတော့
 \begin{itemize}
  \item အနောက်ဘက်ပြန်လှည့်
  \item နံရံဆီကိုသွား 
  \item ညာဘက်လှည့်
  \item ရှေ့တိုး
  \item ညာဘက်လှည့်
\end{itemize}
စသည်ဖြင့် တွေ့နိုင်တယ်။

ဖော်ပြခဲ့သလိုမျိုး အဓိက ဖြေရှင်းရမယ့် အလုပ်\myen{(main task/problem)}ကို ပိုပြီးရိုးရှင်းတဲ့ အလုပ်တွေ\myen{(subtasks/subproblems)} အဖြစ် ခွဲထုတ်မယ်၊ ခွဲထုတ်ရရှိလာတဲ့ အလုပ်တွေကို နောက်ထပ်တစ်ဆင့် ပို၍ပို၍ ရိုးရှင်းတဲ့ အလုပ်တွေဖြစ်အောင် ထပ်ခွဲထုတ်မယ်။ ထိုကဲ့သို့ အလုပ်တစ်ခုကို ရိုးရှင်းသည်ထက် ရိုးရှင်းပြီး ဖြေရှင်းရ လွယ်ကူသည်ထက် လွယ်ကူလာတဲ့ အလုပ်တွေခွဲထုတ်တဲ့ လုပ်ငန်းစဉ်ကို \enProblemDecomposition လို့ခေါ်တယ်။

\section{\myensecttl{Top-Down Approach}}

\enProblemDecomposition နှင့် ဆက်စပ်နေတဲ့ \myen{top-down approach} ဖြင့် \mmprogram ရေးသည့် နည်းလမ်းကို ဆက်လေ့လာကြရအောင်။ ကားရဲလ်ကို တန်းကျော်ပြေး ခိုင်းမယ်။ \mmavenue အရေအတွက်က ၁၄ လမ်း ပုံသေဖြစ်မယ်။ \mmstreet အရေအတွက်က ကမ္ဘာတစ်ခုနဲ့တစ်ခု တူချင်မှ တူမယ်။ တန်းတွေရဲ့ အမြင့်နဲ့ နေရာတွေလည်း မတူဘူးလို့ ယူဆပါ။ နမူနာ ကမ္ဘာတစ်ခုကို \Fig \vref*{fig:HurdleJumping} တွင် တွေ့နိုင်တယ်။ အလားတူ အခြား ကမ္ဘာတစ်ခု အတွက်လည်း ကားရဲလ်ကို တန်းတွေ တစ်ခုချင်းစီကို ကျော်ဖြတ်ပြီး ပန်းတိုင်ဖြစ်တဲ့ (၁၄, ၁) \mmcorner ကို ရောက်အောင် သွားခိုင်းရမှာပါ။ 

\begin{figure}[h]
  \caption{\mymmfigcpt{တန်းကျော်ပြေး ပြိုင်ပွဲဝင်မည့် ကားရဲလ်၏ ကမ္ဘာ}}\label{fig:HurdleJumping}
  \includegraphics[width=4.5in, left]{ch03/HurdleJumping/init_w1.jpg}
\end{figure}

%\begin{lstcodesimple}[float, caption=ပထမစမ်းကြည့်ပုံ, label={lst:HurdleJumpingMain}]
%public class HurdleJumping extends stanford.karel.Karel {
%        public void run(){
%                // Karel has to move or jump over the hurdle 
%                // until (14,1). There 
%                for (int i = 0; i < 13; i++) {
%                        if (frontIsClear()) {
%                                move();
%                        } else {
%                                jumpOverAHurdle();
%                        }
%                }
%        }
%}
%\end{lstcodesimple}

\mmprogram ချမရေးသေးပဲ အဓိကအလုပ်ကို ပိုင်းခြားကြည့်လျှင် တန်းတစ်ခု ကျော်ခြင်းနှင့် ရှေ့တိုးခြင်း* အလုပ်အခွဲ နှစ်ခုကို တွေ့ရတယ်။ ရှေ့ကိုတိုးခိုင်းဖို့က \mycode{move} ကို ကားရဲလ် နားလည်ပြီးသား။ တန်းတစ်တန်း ကျော်ခိုင်းဖို့ \mycode{jumpOverAHurdle} \mmcommand သာရှိခဲ့မယ်၊ ကားရဲလ်သာ ဒီ \mmcommand ကို နားလည်ခဲ့မယ်ဆိုလျှင် 
\begin{lstcodeminimal}
for (int i = 0; i < 13; i++) {
        if (frontIsClear()) {
                move();
        } else {
                jumpOverAHurdle();
        }
}
\end{lstcodeminimal} 
ပုံစံဖြင့် ရေးနိုင်တယ်။ \mycode{jumpOverAHurdle} က တကယ်မရှိသေးတဲ့ အတွက် ဒီ \mmprogram ကို \mmrun လို့တော့ မရသေးပါဘူး။

အခု အဆင့်မှာပဲ ဒီအတိုင်းရေးထားတာ မှန်တယ်ဆိုတာ ရှေ့ဆက် မသွားခင် ဘယ်လို သေချာအောင် လုပ်လို့ရမလဲ။ စမ်းကြည့်လို့ရမလဲ။ \mycode{jumpOverAHurdle} ကလဲ အမှန်တကယ် မရှိသေးဘူး။ တစ်ခု လုပ်လို့ရတာက \mycode{jumpOverAHurdle} \mmcommand က ဘာလုပ်ပေးမှာလဲ တိတိကျကျ အရင်ဆုံး သတ်မှတ်ထားမယ်။ တနည်းအားဖြင့် ဒီ \mmcommand ကို မလုပ်ဆောင်မီနဲ့ လုပ်ဆောင်ပြီး ရှိနေရမယ့် အခြေအနေကို အတိအကျ သတ်မှတ်ထားမယ်လို့ ဆိုလိုတာပါ။ ရွေးချယ် သတ်မှတ်လို့ ရနိုင်တဲ့ အခြေအနေ နှစ်ခုစီကို \Fig \ref*{fig:jumpOverAHurdlePre}   နှင့် \ref*{fig:jumpOverAHurdlePost} တို့တွင်ကြည့်ပါ။ အဆင်ပြေနိုင်တဲ့ တစ်ခုစီကို ရွေးချယ်သတ်မှတ်ပြီး၊ သတ်မှတ်ချက်အတိုင်း \mycode{jumpOverAHurdle} လုပ်ဆောင်ပေးမယ်ဆိုလျှင် အထက်ပါ အဓိက \enForLoop ဘယ်လိုအလုပ်လုပ်မလဲ၊  \mmiteration တစ်ခေါက် ပြီးတိုင်း ဘယ်လို အခြေအနေမှာ ရှိနေမလဲ၊ နောက်ဆုံး တစ်ဆယ်သုံးကြိမ်မြောက် \mmiteration အပြီးမှာ လိုရာပန်းတိုင်ကို ကားရဲလ်ရောက်သွား မှာလား စသည်ဖြင့် တဆင့်ချင်း စစ်ကြည့်လို့ရပါတယ်။

\begin{figure}[tbh!]
  \caption{\myenlstcpt{\mycodefigcpt{jumpOverAHurdle}} \mymmfigcpt{မလုပ်ဆောင်မီ ရှိနေနိုင်သည့် အခြေအနေ နှစ်ခု}}
  \begin{subfigure}[t]{0.3\textwidth}
      \includegraphics[width=1in, left]{ch03/HurdleJumping/JumpOverPre1.jpg}
      \caption{}
      \label{fig:JumpOverPre1}
  \end{subfigure}
  \begin{subfigure}[t]{0.3\textwidth}
      \includegraphics[width=1in, left]{ch03/HurdleJumping/JumpOverPre2.jpg}
      \caption{}
  \end{subfigure}
  \label{fig:jumpOverAHurdlePre}
\end{figure}

\begin{figure}[tbh!]
  \caption{\mycodefigcpt{jumpOverAHurdle} \mymmfigcpt{လုပ်ဆောင်ပြီး ရှိနေနိုင်သည့် အခြေအနေ နှစ်ခု}}
  \begin{subfigure}[t]{0.3\textwidth}
    \includegraphics[width=1in, left]{ch03/HurdleJumping/JumpOverPost1.jpg}
    \caption{}
    \label{fig:JumpOverPost1}
  \end{subfigure}
  \begin{subfigure}[t]{0.3\textwidth}
      \includegraphics[width=1in, left]{ch03/HurdleJumping/JumpOverPost2.jpg}
      \caption{}
  \end{subfigure}
  \label{fig:jumpOverAHurdlePost}
\end{figure}

\Fig \ref*{fig:JumpOverPre1} နှင့် \ref*{fig:JumpOverPost1} ကို မလုပ်ဆောင်မီနှင့် လုပ်ဆောင်ပြီး အခြေအနေတွေအဖြစ် ရွေးချယ်သတ်မှတ်တယ်လို့ ယူဆပြီး \mmiteration တစ်ခါပြီးတိုင်း ရှိနေမည့် အခြေအနေကို လိုက်ကြည့်လျှင် \Fig \ref{fig:HurdleJumpingIters} မှာပြထားတဲ့ အတိုင်း တွေ့ရမယ်။ \enForLoop ပထမ တစ်ကျော့မှာ \mycode{move();}  ဒုတိယ တစ်ကျော့တွင် \mycode{jumpOverAHurdle();} လုပ်ဆောင်မည်ဖြစ်ပြီး \Fig \ref*{fig:HurdleJumping1stIter} နှင့် \ref*{fig:HurdleJumping2ndIter} တို့သည် သတ်မှတ်ထားသော \mycode{jumpOverAHurdle} မလုပ်ဆောင်မီနှင့် လုပ်ဆောင်ပြီး အခြေနေတို့ဖြင့် ကိုက်ညီကြောင်း သတိပြုပါ။ ထို့အတူ တတိယ အကျော့မှာ \mycode{move();}  စတုတ္ထ အကျော့တွင် \mycode{jumpOverAHurdle();}ကို လုပ်ဆောင်မည်ဖြစ်ပြီး \Fig \ref*{fig:HurdleJumping3rdIter} နှင့် \Fig \ref*{fig:HurdleJumping4thIter} တို့သည် သတ်မှတ်ထားသော \mycode{jumpOverAHurdle} မလုပ်ဆောင်မီနှင့် လုပ်ဆောင်ပြီး အခြေနေတို့ဖြင့် ကိုက်ညီတယ်။
\begin{figure}[!htb]
  \caption{\mymmfigcpt{\mmiteration လေးခု}}
  \begin{subfigure}[t]{0.3\textwidth}
      \adjincludegraphics[height=2in,trim={0 0 {.61\width} 0}, clip, left]{ch03/HurdleJumping/init_w1.jpg}
      \caption{}
      \label{fig:HurdleJumpingInit}
  \end{subfigure}
  \begin{subfigure}[t]{0.3\textwidth}
    \adjincludegraphics[height=2in,trim={0 0 {.61\width} 0}, clip, left]{ch03/HurdleJumping/1st_iter.jpg}
      \caption{}
      \label{fig:HurdleJumping1stIter}
  \end{subfigure}
  \begin{subfigure}[t]{0.3\textwidth}
    \adjincludegraphics[height=2in,trim={0 0 {.61\width} 0}, clip, left]{ch03/HurdleJumping/2nd_iter.jpg}
    \caption{}
    \label{fig:HurdleJumping2ndIter}
\end{subfigure}
\hspace{0.1in}
\begin{subfigure}[t]{0.3\textwidth}
  \adjincludegraphics[height=2in,trim={0 0 {.61\width} 0}, clip, left]{ch03/HurdleJumping/3rd_iter.jpg}
    \caption{}
    \label{fig:HurdleJumping3rdIter}
\end{subfigure}
\begin{subfigure}[t]{0.3\textwidth}
  \adjincludegraphics[height=2in,trim={0 0 {.61\width} 0}, clip, left]{ch03/HurdleJumping/4th_iter.jpg}
    \caption{}
    \label{fig:HurdleJumping4thIter}
\end{subfigure}
\label{fig:HurdleJumpingIters}
\end{figure}

ဒီအတိုင်းသာ \mycode{jumpOverAHurdle}  အလုပ်လုပ်မယ်ဆိုလျှင် ရှေ့မှာ တွေ့ခဲ့တဲ့ \enForLoop အရ ကားရဲလ်ဟာ တန်းတွေကျာ်ပြီး ပန်းတိုင်ရောက်မှာပါ။ ဒါပေမယ့် \mycode{jumpOverAHurdle} က အမှန်တကယ် မရှိသေးပါဘူး။ ဒီ\mmcommand\ ဘာလုပ်ပေးမလဲ ဆိုတာကိုပဲ တိတိကျကျ သတ်မှတ်ပြီး \enForLoop တစ်ကျော့ပြီးတစ်ကျော့ ရှိနေမယ့် အခြေအနေကို လိုက်ကြည့်ပြီး \mmprogram မှန်မှန်ကန်ကန် အလုပ် လုပ်/မလုပ် စစ်ကြည့်ခဲ့တာပါ။ 

ဒုတိယ အဆင့်မှာ \mycode{jumpOverAHurdle} ကို ဘယ်လိုဖြေရှင်းမလဲ ဆက်ပြီးစဉ်းစားကြမယ်။ အဓိကဖြစ်တဲ့ တန်းအားလုံးအား ကျော်ခြင်း အလုပ်ကို အလုပ်ခွဲတွေအဖြစ် ပိုင်းခြားကြည့်ခဲ့သလိုပဲ တန်းတစ်ခုအား ကျော်ခြင်း အလုပ်ခွဲကိုလည်း အပေါ်တက်ခြင်းနဲ့ အောက်ဆင်းခြင်း ဆိုပြီး ထပ်၍ ပိုင်းခြား ကြည့်လို့ရပါတယ်။ ဒီအလုပ်အခွဲ နှစ်ခုကို လုပ်ဆောင်ပေးမယ့် \mmcommand တွေကို \mycode{ascend} နဲ့ \mycode{descend} လို့ အမည်ပေးမယ်။ ဒီ \mmcommand နှစ်ခုကိုလည်း ရှိပြီးသကဲ့သို့ မှတ်ယူမယ်။ ဘယ်လို လုပ်ဆောင်မှာလဲ မစဉ်းစားသေးပဲ ဘာလုပ်ပေးမှာလဲ အရင်သတ်မှတ်ပါမယ်။
\begin{figure}[!htb]
  \caption{\myenlstcpt{\mycodefigcpt{ascend}} \mymmfigcpt{မလုပ်ဆောင်မီနှင့် လုပ်ဆောင်ပြီး}}
  \begin{subfigure}[t]{0.3\textwidth}
      \includegraphics[width=1in, left]{ch03/HurdleJumping/ascendPre.jpg}
      \caption{}
  \end{subfigure}
  \begin{subfigure}[t]{0.3\textwidth}
      \includegraphics[width=1in, left]{ch03/HurdleJumping/ascendPost.jpg}
      \caption{}
  \end{subfigure}
  \label{fig:ascendPreAndPost}
\end{figure}
\begin{figure}[!htb]
  \caption{\myenlstcpt{\mycodefigcpt{descend}} \mymmfigcpt{မလုပ်ဆောင်မီနှင့် လုပ်ဆောင်ပြီး}}
  \begin{subfigure}[t]{0.3\textwidth}
    \includegraphics[width=1in, left]{ch03/HurdleJumping/descendPre}
    \caption{}
  \end{subfigure}
  \begin{subfigure}[t]{0.3\textwidth}
      \includegraphics[width=1in, left]{ch03/HurdleJumping/descendPost}
      \caption{}
  \end{subfigure}
  \label{fig:descendPreAndPost}
\end{figure}
\Fig \ref*{fig:ascendPreAndPost}   နှင့် \ref*{fig:descendPreAndPost} တို့တွင် ပြထားသလို သတ်မှတ်မယ်။ ဒီအတိုင်းသတ်မှတ်မှ ရမယ်လို့ မဆိုလိုဘူး။ စိတ်ကြိုက် အဆင်ပြေမယ့် အနေအထားကို ရွေးချယ် သတ်မှတ်နိုင်ပါတယ်။ ပုံသေမရှိပါဘူး။ သတ်မှတ်ထားတဲ့ အတိုင်း လုပ်ဆောင်ပေးမယ်လို့ ယူဆပြီး \mycode{jumpOverAHurdle} ကို အခုလို ရေးနိုင်တယ်။ 
\begin{lstcodeminimal}
private void jumpOverAHurdle(){
        ascend(); 
        move(); //// Required!
        descend();
}
\end{lstcodeminimal}
\mycode{ascend} အပြီးမှာ \mycode{descend} မလုပ်ဆောင်မီ ရှိနေရမယ့် သတ်မှတ်ထားတဲ့ အခြေအနေနဲ့ ကိုက်ညီဖို့အတွက် အတွက်  \mycode{move();} \mmcommand လိုအပ်တာကို သတိပြုပါ။ 

အခုဆက်ပြီး ဖြေရှင်းရမှာက \mycode{ascend} နဲ့ \mycode{descend} ပါ။ ဒီအလုပ် နှစ်ခုကို ပိုပြီး ရိုးရှင်းတဲ့ အလုပ်တွေဖြစ်အောင် ထပ်ပြီး ခွဲခြမ်းကြည့်လို့ ရမလား သို့ မဟုတ် ခွဲခြမ်းကြည့်သင့် မလား  စဉ်းစားရမယ်။ ဘယ်လောက်ရိုးရှင်းတဲ့ အလုပ်ခွဲတွေ ဖြစ်လာတဲ့ အထိ ဒီအတိုင်းဆက်ပြီး ခွဲနေရမလဲ။ ဘယ်အခါမှာ အလုပ်အခွဲ အသစ်တွေ ရှာနေတာကို ရပ်ရမလဲ။ ဒါ‌တွေကလဲ စဉ်းစား ရမယ့် အချက်တွေပါ။ 

လက်ရှိအဆင့်မှာ အလုပ်အခွဲကို \mmcommand တစ်ခု အနေနဲ့ ရေးရတာ မခက်တော့ဘူး၊ အတန်အသင့် လွယ်လွယ်ကူကူ ရေးလို့ရပြီဟု ယူဆလျှင် သော်လည်းကောင်း၊ လက်ရှိ ထပ်မံ ပိုင်းခြားရန် စဉ်းစားမည့် အလုပ်အခွဲကို ဖွဲ့စည့်ထားသည့် ပို၍ရိုးရှင်းသော အလုပ်အခွဲငယ်များ၏ အမည်ကို စဉ်းစား၍ မရတော့လျှင် သော်လည်းကောင်း ရပ်လိုက်နိုင်သည်ဟု အကြမ်းဖျဉ်း ပြောနိုင်တယ်။ 

\mycode{jumpOverAHurdle} မှာတော့ တက်တာနဲ့ ဆင်းတာ အလုပ်နှစ်ခုပါတယ်ဆိုတာ မြင်သာတယ်။ အတက်နဲ့ အဆင်း အမည်ပေးရတာလဲ သိပ်စဉ်းစားနေစရာ မလိုဘူး။ \mycode{ascend} နဲ့ \mycode{descend} ကို ဖွဲ့စည်းထားတဲ့ အလုပ်ခွဲတွေက ဘာတွေဖြစ်မလဲ။ သူတို့ထက် ပိုပြီး ရိုးရှင်းတဲ့ အလုပ်က ဘယ်ဘက်လှည့်တာ၊ ညာဘက်လှည့်တာ ရှိမယ်။ ဘယ်ဘက်လှည့်ဖို့ အတွက်က \mycode{turnLeft} ရှိထားပြီးပြီ။ ညာဘက်လှည့်ဖို့ အတွက် \mycode{turnRight} အလုပ်အခွဲ သတ်မှတ်နိုင်တယ်။ တန်းတစ်ခု ထိပ်ထိရောက်အောင် သွားဖို့အတွက် \mycode{while(rightIsBlocked())} သုံးပြီး သွားလို့ရတယ်ဆိုတာ ရှေ့အခန်းမှာ တွေ့ခဲ့ပြီးပြီ။ သိပ်လဲမခက်ဘူး။ ဖော်ပြပါ အချက်တွေကြောင့် အတက်နဲ့ အဆင်း အလုပ်နှစ်ခုကို ထပ်ပြီီးမခွဲတော့ဖို့ ဆုံးဖြတ်နိုင်ပြီး အောက်ပါအတိုင်း ရေးနိုင်တယ်။
\begin{lstcodeminimal}
private void ascend() {
        turnLeft();
        while (rightIsBlocked()) {
                move();
        }
        turnRight();
}
private void descend() {
        turnRight();
        while (frontIsClear()) {
                move();
        }
        turnLeft();
}
\end{lstcodeminimal}
ဒီအခါမှာ တန်းတွေကို တစ်ခုချင်းစီ ကျော်ပြီး ကားရဲလ်ကို ပန်းဝင်အောင် သွားခိုင်းဖို့ \mmprogram တစ်ခုလုံးလည်း ရေးပြီးသွားပါပြီ။ အပြည့်အစုံကို \Lst \ref{lst:HurdleJumpingAscDscImpl} တွင် ‌လေ့လာ ကြည့်ပါ။

\begin{lstcodelong}[{\mycodelstcpt{HurdleJumping}}, label={lst:HurdleJumpingAscDscImpl}] 
public class HurdleJumping extends stanford.karel.Karel {
        public void run(){
                for (int i = 0; i < 13; i++) {
                        if (frontIsClear()) {
                                move();
                        } else {
                                jumpOverAHurdle();
                        }
                }
        }

        private void jumpOverAHurdle(){
                ascend();
                move(); // Required!
                descend();
        }

        private void ascend() {
                turnLeft();
                while (rightIsBlocked()) {
                        move();
                }
                turnRight();
        }

        private void descend() {
                turnRight();
                while (frontIsClear()) {
                        move();
                }
                turnLeft();
        }

        private void turnRight() {
                turnLeft();
                turnLeft();
                turnLeft();
        }
}
\end{lstcodelong}

%\begin{lstcodesimple}[caption=ပထမစမ်းကြည့်ပုံ, label={lst:HurdleJumpingJumpHurdleImpl}] 
%public class HurdleJumping extends stanford.karel.Karel {
%        public void run(){
%                for (int i = 0; i < 13; i++) {
%                        if (frontIsClear()) {
%                                move();
%                        } else {
%                                jumpOverAHurdle();
%                        }
%                }
%        }
%        
%        private void jumpOverAHurdle(){
%                ascend();
%                // move(); is required due to the specified pre
%                // and post condition of ascend and descend 
%                move();
%                descend();
%        }
%}
%\end{lstcodesimple}

\mycode{HurdleJumping} \mmprogram တစ်ခုလုံး ပြီးတဲ့အထိ တဆင့်ချင်းလုပ်ခဲ့တာတွေကို ပြန်အနှစ်ချုပ် ကြည့်လျှင် အောက်ပါ   အတိုင်းတွေ့ရမယ်။ 

\begin{enumerate}
  \item တန်းအားလုံးကျော်ပြီး ပန်းတိုင်ရောက်အောင် သွားဖို့အတွက် အဓိကအလုပ်မှာ ပါဝင်တဲ့ အလုပ်အခွဲတွေ ကို စဉ်းစားတယ်။ တန်းတစ်ခု ကျော်တာသည် အလုပ်အခွဲ  ဖြစ်တယ်။  ဒီအလုပ်အခွဲကို လုပ်ဆောင်ပေးမယ့် \mmcommand ကို \mycode{jumpOverAHurdle} အမည်ပေးတယ်။ \mycode{jumpOverAHurdle} မလုပ်ဆောင်မီ ရှိရမည့် အခြေအနေနဲ့ လုပ်ဆောင်ပြီး ရှိရမည့် အခြေအနေတို့ကို သတ်မှတ်တယ်။ ဒီသတ်မှတ်ချက်တွေ အတိုင်း လုပ်ဆောင်မယ်လို့ ယူဆပြီး တန်းအားလုံးကျော်ပြီး ပန်းတိုင်ရောက်အောင် သွားဖို့အတွက်ကို ဖြေရှင်းတယ်။     \label{itm:HurdleJumpingMain}
  
  \item \mycode{jumpOverAHurdle} က ဘယ်လိုဖြေရှင်းရမလဲ ဆက်ပြီးစဉ်းစားတယ်။ တန်းတစ်ခု ကျော်တဲ့ အလုပ်ကိုလဲ အလုပ်အခွဲတွေ အဖြစ်ထပ်ပြီး ခွဲတယ်။ အပေါ်တက်နဲ့ အောက်ဆင်း နှစ်ခု ပါဝင်တာတွေ့ရတယ်။ ဒီနှစ်ခုကို လုပ်ဆောင်ပေးမယ့် \mmcommand တွေကို \mycode{ascend} နဲ့ \mycode{descend} အမည်ပေးတယ်။ \mycode{ascend} နဲ့ \mycode{descend} မလုပ်ဆောင်မီ ရှိရမည့် အခြေအနေနဲ့ လုပ်ဆောင်ပြီး ရှိရမည့် အခြေအနေတို့ကို သတ်မှတ်တယ်။ သတ်မှတ်ချက်အတိုင်း လုပ်ဆောင်မယ်လို့ ယူဆပြီး \mycode{jumpOverAHurdle} ဖြေရှင်းတယ်။     \label{itm:HurdleJumpingSub}

  \item အပေါ်တက်တဲ့ အလုပ်နဲ့ အောက်ဆင်းတဲ့ အလုပ်ကို ထပ်ပြီးခွဲကြည့်ဖို့ စဉ်းစားတယ်။ သို့ပေမယ့် အဓိပ္ပါယ်ရှိတဲ့ အလုပ်အခွဲကို နာမည်ပေးဖို့ စဉ်းစားလို့မရတော့ဘူး။ ဒါ့အပြင် ဒီအလုပ်နှစ်ခုကို \mmprogram ရေးဖို့ အတွက် အတန်အသင့် လွယ်ကူမယ်လို့လဲ ယူဆတယ်။ ဒါကြောင့် ဒီနှစ်ခုကို ဆက်မခွဲတော့ပဲ\mycode{ascend} နဲ့ \mycode{descend} တွေကို သတ်မှတ်တယ်။ 
  \label{itm:HurdleJumpingJumpSubsub}
\end{enumerate}
%\bigskip

ဖော်ပြခဲ့သလို အဓိက အလုပ်ကို ဖြေရှင်းဖို့အတွက် အလုပ်အခွဲတွေ အဖြစ်ပိုင်းခြားပြီး ရရှိလာတဲ့ အလုပ်အခွဲတွေကို လုပ်ဆောင်ပေးမယ့် \mmcommand တွေ ရှိထားပြီး ဖြစ်သကဲ့သို့ မှတ်ယူပြီး (ဒီ \mmcommand တွေဘာ လုပ်ပေးမှာလဲ၊ မလုပ်ဆောင်မီနှင့် လုပ်ဆောင်ပြီး အခြေအနေ တွေကို သတ်မှတ်ပြီး) အဓိက အလုပ်ကို ဖြေရှင်းတဲ့ နည်းလမ်းကို \mmTopDown နည်းလမ်းလို့ ခေါ်တယ်။  \enStepWiseRefinement ဟုလည်း ခေါ်တယ်။ 

ဖော်ပြခဲ့သလိုမျိုး လုပ်ငန်းစဉ်အတိုင်း နောက်ထပ် ပုစ္ဆာတစ်ခုကို ဖြေရှင်းကြည့်ရအောင်။ \Fig \ref{fig:SweepTheStreetsW1} နမူနာ ကမ္ဘာကို ပြထားတယ်။ 
\begin{figure}[!htb]
  \caption{ကားရဲလ်နှင့်  ကားရဲလ်၏ကမ္ဘာ}\label{fig:SweepTheStreetsW1}
  \includegraphics[width=4in, left]{ch03/SweepTheStreets/init_w1.jpg}
\end{figure}
လုပ်ရမယ့် အလုပ်က \mmbeeper တွေ ကုန်အောင် ရှင်းပေးရမှာပါ။ အလားတူ မည်သည့် $m \times n$ ကမ္ဘာတွင်မဆို အမှိုက်တွေ ကုန်အောင် ရှင်းပေးရမှာပါ။

အခန်းတစ်ခန်းလုံး ရှင်းဖို့အလုပ်မှာ တစ်လမ်းရှင်းတာ၊ မြောက်ဘက် လှည့်တာ၊ နောက်လမ်းတစ်ခု ဆက်ရှင်းဖို့အတွက် အဆင်သင့်ပြင်တာ စတဲ့ အလုပ်အခွဲ သုံးခု ပါဝင်တယ်။ ဒီအလုပ် သုံးခုကို
\begin{enumerate}
  \item \mycode{sweepOneSt}
  \item \mycode{turnNorth}
  \item \mycode{getReadyForNextSt}
\end{enumerate}
\mmcommand တွေအနေနဲ့ အသီးသီး ယူဆမယ်။ ဒီသုံးခုက ဘာလုပ်ပေးမှာလဲ အရင်ဆုံး သတ်မှတ်မယ်။ 
\begin{enumerate}
\item \mycode{sweepOneSt}
\begin{itemize}
\item လမ်းတစ်လမ်း၏ အစ သို့ အဆုံးတွင် အခြားဘက်စွန်းသို့ မျက်နှာမူပြီး ရှိနေမယ်။
\item လမ်း၏ အခြားဘက်အဆုံးသို့ ရောက်ရှိပြီး ရှေ့တွင် နံရံပိတ်နေမယ်။ \mmbeeper အားလုံး ရှင်းပြီး ဖြစ်မယ်။ \Fig \ref{fig:sweepOneStPreAndPostConds} တွင်ကြည့်ပါ။
\end{itemize}
\item \mycode{turnNorth}  
  \begin{itemize}
    \item \mmcorner တစ်ခုတွင် ရှိနေမည်။
    \item မည်သည့် အရပ်သို့ မျက်နှာမူနေသည်ဖြစ်စေ မြောက်ဘက်သို့ လှည့်ပြီးဖြစ်နေရမည်။
  \end{itemize} 
  \item \mycode{getReadyForNextSt}
  \begin{itemize}
    \item လမ်းတစ်လမ်း၏ အစ သို့ အဆုံးတွင် မြောက်ဘက်သို့ မျက်နှာမူပြီးရှိနေမည်။
    \item အပေါ်တစ်လမ်းကူးပြီး လမ်း၏ အခြားဘက်ဆုံးသို့ မျက်နှာမူပြီးရှိနေမည်။ \Fig \ref{fig:getReadyForNextPostAndPost} တွင်ကြည့်ပါ။
  \end{itemize}   
\end{enumerate}
\mmcommand သုံးခုက ဒီသတ်မှတ်ချက်တွေအတိုင်း အလုပ်လုပ်မယ်ဆိုရင် လမ်းအားလုံးရှင်းဖို့ အတွက် အခုလို ရေးလို့ရမယ်။ 
\begin{lstcodeminimal}
sweepOneSt();
while(frontIsClear()) {
        getReadyForNextSt();
        sweepOneSt();
}
\end{lstcodeminimal}

%\begin{lstcodesimple}[float=tbh!, caption=ပထမစမ်းကြည့်ပုံ, label={lst:SweepTheStreetsMain}] 
%public class SweepTheStreets extends stanford.karel.Karel {
%        public void run() {
%                cleanOneSt();
%                while(frontIsClear()) {
%                        getReadyForNextSt();
%                        cleanOneSt();
%                }
%        }
%}
%\end{lstcodesimple}

\begin{figure}[tbh!]
  \caption{\mymmfigcpt{\mmiteration လေးခု}}
  \begin{subfigure}[t]{0.48\textwidth}
      \adjincludegraphics[width=2.5in,trim={0 0 0 {.55\height}}, clip, left]{ch03/SweepTheStreets/init_w1.jpg}
      \caption{}
      \label{fig:sweepOneStPreOdd}
  \end{subfigure}
  \hspace{0.1in}
  \begin{subfigure}[t]{0.48\textwidth}
    \adjincludegraphics[width=2.5in,trim={0 0 0 {.55\height}}, clip, left]{ch03/SweepTheStreets/sweepOneStPostOdd.jpg}
      \caption{}
      \label{fig:sweepOneStPostOdd}
  \end{subfigure}

  \vspace{0.1in}
  \begin{subfigure}[t]{0.48\textwidth}
    \adjincludegraphics[width=2.5in,trim={0 0 0 {.55\height}}, clip, left]{ch03/SweepTheStreets/sweepOneStPreEven.jpg}
    \caption{}
    \label{fig:sweepOneStPreEven}
  \end{subfigure}
  \hspace{0.1in}
  \begin{subfigure}[t]{0.48\textwidth}
    \adjincludegraphics[width=2.5in,trim={0 0 0 {.55\height}}, clip, left]{ch03/SweepTheStreets/sweepOneStPostEven.jpg}
      \caption{}
      \label{fig:sweepOneStPostEven}
  \end{subfigure}
  \label{fig:sweepOneStPreAndPostConds}
\end{figure}

\begin{figure}[tbh!]
  \caption{\mymmfigcpt{\mmiteration လေးခု}}
  \begin{subfigure}[t]{0.48\textwidth}
      \adjincludegraphics[width=2.5in,trim={0 0 0 {.4\height}}, clip, left]{ch03/SweepTheStreets/getReadyForNextPreOdd.jpg}
      \caption{}
      \label{fig:getReadyForNextPreOdd}
  \end{subfigure}
  \hspace{0.1in}
  \begin{subfigure}[t]{0.48\textwidth}
    \adjincludegraphics[width=2.5in,trim={0 0 0 {.4\height}}, clip, left]{ch03/SweepTheStreets/getReadyForNextPostOdd.jpg}
      \caption{}
      \label{fig:getReadyForNextPostOdd}
  \end{subfigure}

  \vspace*{0.1in}
  \begin{subfigure}[t]{0.48\textwidth}
    \adjincludegraphics[width=2.5in,trim={0 0 0 {.4\height}}, clip, left]{ch03/SweepTheStreets/getReadyForNextPreEven.jpg}
    \caption{}
    \label{fig:getReadyForNextPreEven}
  \end{subfigure}
  \hspace{0.1in}
  \begin{subfigure}[t]{0.48\textwidth}
    \adjincludegraphics[width=2.5in,trim={0 0 0 {.4\height}}, clip, left]{ch03/SweepTheStreets/getReadyForNextPostEven.jpg}
      \caption{}
      \label{fig:getReadyForNextPostEven}
  \end{subfigure}
  \label{fig:getReadyForNextPostAndPost}
\end{figure}

\noindent ဒုတိယ  အဆင့်မှာ 
\begin{enumerate}
  \item \mycode{sweepOneSt} \label{itm:sweepOneSt}
  \item \mycode{turnNorth} \label{itm:turnNorth}
  \item \mycode{getReadyForNextSt} \label{itm:getReadyForNextSt}
\end{enumerate}
တို့ကို ဆက်ပြီး ဖြေရှင်းရမယ်။ \mycode{sweepOneSt} ကိုခွဲကြည့်မယ် ဆိုလျှင် \mmcorner တစ်ခုကို တံမြက်စည်း စင်အောင်လှည်းတာ ပါဝင်မယ်။ ဒီအလုပ်ကို \mycode{sweepOneCnr} ဟု အမည်ပေးရအောင်။ \mycode{turnNorth} ကိုတော့ အတန်အသင့် လွယ်ကူတဲ့အတွက် ထပ်မခွဲတော့ဘူး။ \mycode{getReadyForNextSt} ကိုတော့ အနောက်ဘက် လှည့်ခိုင်းခြင်း၊ အရှေ့ဘက် လှည့်ခိုင်းခြင်း ပါဝင်နေတယ်လို့ ယူဆနိုင်ပြီး \mycode{turnEast} နဲ့ \mycode{turnWest} လို့ အမည်ပေးမယ်။ *။ မလုပ်ဆောင်မီ နှင့် လုပ်ဆောင်ပြီး အခြေအနေတွေက ပေးထားတဲ့ အမည်ကနေတဆင့် သိသာနေပြီး ဖြစ်တဲ့အတွက် မဖော်ပြတော့ပဲ \Lst \ref{lst:SweepTheStreets} \mmprogram တစ်ခုလုံး အပြည့်အစုံ တွေ့နိုင်ပါတယ်။ 

\begin{lstcodelong}[{\mycodelstcpt{SweepTheStreets}}, label={lst:SweepTheStreets}] 
public class SweepTheStreets extends stanford.karel.Karel {
        public void run() {
                sweepOneSt();
                while(frontIsClear()) {
                        getReadyForNextSt();
                        sweepOneSt();
                }
        }

        private void sweepOneSt() {
                sweepOneCnr();
                while (frontIsClear()) {
                        move();
                        sweepOneCnr();
                }
                turnNorth();
        }

        private void sweepOneCnr() {
                while (beepersPresent()) {
                        pickBeeper();
                }
        }

        private void turnNorth() {
                while (notFacingNorth()) {
                        turnLeft();
                }
        }

        private void getReadyForNextSt() {
                move();
                if (rightIsBlocked()) {
                        turnWest();
                }
                if (leftIsBlocked()) {
                        turnEast();
                }
        }

        private void turnWest() {
                while(notFacingWest()) {
                        turnLeft();
                }
        }

        private void turnEast() {
                while(notFacingEast()) {
                        turnLeft();
                }
        }
}
\end{lstcodelong}

\mmcorner တစ်ခုမှာ စုပုံထားတဲ့ \mmbeeper တွေကို နှစ်ဆဖြစ်သွားအောင် ကားရဲလ်ကို လုပ်ခိုင်းချင်တယ်။ ကားရဲလ်က သင်္ချာတော့ အားနည်းပါတယ်။ ရေတွက်တာ \myen{(counting)} နဲ့ ပေါင်းနှုတ်မြှောက်စား စတာတွေ နားမလည်ဘူး။ မှတ်ထားနိုင်တဲ့ မှတ်ညာဏ်လည်း သူ့မှာမရှိဘူး။ အခြားနည်းလမ်း ရှာပြီးခိုင်းရမယ်။ ပုံထားတဲ့ နေရာက \mmbeeper တစ်ခုချင်းစီကို ကောက်ပြီး ရှေ့ \mmcorner မှာ နှစ်ခုပြန်ချ*။ ဒါကို နကိုနေရာက  \mmbeeper တွေကုန်တဲ့ အထိ လုပ်လျှင် နကို ပုံထားတဲ့ နေရာရှေ့တစ် \mmcorner မှာ နှစ်ဆပိုများတဲ့ \mmbeeper တွေ ပုံးထားပြီးဖြစ်နေမယ်။ အဲဒီ \mmbeeper တွေကို နကိုနေရာကို ပြန်ရွှေ့လိုက်ရင် ခိုင်းချင်တဲ့ အလုပ်ပြီးပါပြီ။ ပထမအဆင့် အလုပ်တွေကို ခွဲကြည့်လျှင်

\begin{enumerate}
  \item \mycode{goToBeeperPile}
  \item \mycode{doubleTheBeeperPileAtNxtCnr}
  \item \mycode{moveTheBeeperPileAtNxtCnr}
\end{enumerate}
\mmcommand တွေအနေနဲ့ အသီးသီး ယူဆမယ်။ ဒီသုံးခုက ဘာလုပ်ပေးမှာလဲ အရင်ဆုံး သတ်မှတ်မယ်။ 
\begin{enumerate}
  \item \mycode{goToBeeperPile}
  \begin{itemize}
    \item (၁, ၁) \mmcorner တွင် အရှေ့ဘက် မျက်နှာမူပြီး ရှိနေမယ်။
    \item \mmbeeper များပုံထားသည့် \mmcorner တွင် အရှေ့ဘက်သို့ မျက်နှာမူပြီး ရှိနေမယ်။။ \Fig \ref{fig:goToBeeperPilePreAndPost} တွင်ကြည့်ပါ။
  \end{itemize}
  \item \mycode{doubleTheBeeperPileAtNxtCnr}  
    \begin{itemize}
      \item \mmbeeper များ ပုံထားသည့် \mmcorner  တွင်ရှိနေသည်။
      \item ရှေ့တစ် \mmcorner တွင် \mmbeeper အရေအတွက် နှစ်ဆရှိနေပြီး အရှေ့ဘက်သို့ မျက်နှာမူ၍ နေမည်။ \Fig \ref{fig:doubleTheBeeperPileAtNxtCnrPreAndPost} တွင်ကြည့်ပါ။
    \end{itemize} 
  \item \mycode{moveTheBeeperPileAtNxtCnr}
    \begin{itemize}
      \item \mmbeeper များ ပုံထားသည့် \mmcorner  တွင် နောက်ဘက်သို့ မျက်နှာမူ၍ ရှိနေသည်။
      \item ရှေ့တစ် \mmcorner တွင် \mmbeeper များရောက်သွားပြီး အနောက်ဘက်သို့ မျက်နှာမူ၍ ရှိနေမည်။  \Fig \ref{fig:moveTheBeeperPileAtNxtCnrPreAndPost} တွင်ကြည့်ပါ။
    \end{itemize}   
\end{enumerate}

\begin{figure}[tbh!]
  \caption{\mymmfigcpt{\mycodefigcpt{goToBeeperPile} \mymmfigcpt{မလုပ်ဆောင်မီ က) နှင့် လုပ်ဆောင်ပြီး ခ)}}}
  \begin{subfigure}[t]{0.48\textwidth}
      \adjincludegraphics[width=2.5in,trim={0 0 0 {.4\height}}, clip, left]{ch03/DoubleBeeperPile/goToBeeperPilePre.jpg}
      \caption{}
      \label{fig:goToBeeperPilePre}
  \end{subfigure}
  \hspace{0.1in}
  \begin{subfigure}[t]{0.48\textwidth}
    \adjincludegraphics[width=2.5in,trim={0 0 0 {.4\height}}, clip, left]{ch03/DoubleBeeperPile/goToBeeperPilePost.jpg}
      \caption{}
      \label{fig:goToBeeperPilePost}
  \end{subfigure}
  \label{fig:goToBeeperPilePreAndPost}
\end{figure}
\begin{figure}[tbh!]
  \caption{\mymmfigcpt{\mycodefigcpt{doubleTheBeeperPileAtNxtCnr} \mymmfigcpt{မလုပ်ဆောင်မီ က) နှင့် လုပ်ဆောင်ပြီး ခ)}}}
  \begin{subfigure}[t]{0.48\textwidth}
      \adjincludegraphics[width=2.5in,trim={0 0 0 {.4\height}}, clip, left]{ch03/DoubleBeeperPile/doubleTheBeeperPileAtNxtCnrPre.jpg}
      \caption{}
      \label{fig:doubleTheBeeperPileAtNxtCnrPre}
  \end{subfigure}
  \hspace{0.1in}
  \begin{subfigure}[t]{0.48\textwidth}
    \adjincludegraphics[width=2.5in,trim={0 0 0 {.4\height}}, clip, left]{ch03/DoubleBeeperPile/doubleTheBeeperPileAtNxtCnrPost.jpg}
      \caption{}
      \label{fig:doubleTheBeeperPileAtNxtCnrPost}
  \end{subfigure}
  \label{fig:doubleTheBeeperPileAtNxtCnrPreAndPost}
\end{figure}
\begin{figure}[tbh!]
  \caption{\mymmfigcpt{\mycodefigcpt{moveTheBeeperPileAtNxtCnr} \mymmfigcpt{မလုပ်ဆောင်မီ က) နှင့် လုပ်ဆောင်ပြီး ခ)}}}
  \begin{subfigure}[t]{0.48\textwidth}
      \adjincludegraphics[width=2.5in,trim={0 0 0 {.4\height}}, clip, left]{ch03/DoubleBeeperPile/moveTheBeeperPileAtNxtCnrPre.jpg}
      \caption{}
      \label{fig:moveTheBeeperPileAtNxtCnrPre}
  \end{subfigure}
  \hspace{0.1in}
  \begin{subfigure}[t]{0.48\textwidth}
    \adjincludegraphics[width=2.5in,trim={0 0 0 {.4\height}}, clip, left]{ch03/DoubleBeeperPile/moveTheBeeperPileAtNxtCnrPost.jpg}
      \caption{}
      \label{fig:moveTheBeeperPileAtNxtCnrPost}
  \end{subfigure}
  \label{fig:moveTheBeeperPileAtNxtCnrPreAndPost}
\end{figure}

\begin{figure}[tbh!]
  \caption{\mymmfigcpt{\mycodefigcpt{moveOneBeeperAtNxtCnr} \mymmfigcpt{မလုပ်ဆောင်မီ က) နှင့် လုပ်ဆောင်ပြီး ခ)}}}
  \begin{subfigure}[t]{0.48\textwidth}
      \adjincludegraphics[width=2.5in,trim={0 0 0 {.4\height}}, clip, left]{ch03/DoubleBeeperPile/moveOneBeeperAtNxtCnrPre.jpg}
      \caption{}
      \label{fig:moveOneBeeperAtNxtCnrPostPre}
  \end{subfigure}
  \hspace{0.1in}
  \begin{subfigure}[t]{0.48\textwidth}
    \adjincludegraphics[width=2.5in,trim={0 0 0 {.4\height}}, clip, left]{ch03/DoubleBeeperPile/ moveOneBeeperAtNxtCnrPost.jpg}
      \caption{}
      \label{fig:moveOneBeeperAtNxtCnrPost}
  \end{subfigure}
  \label{fig:moveOneAtNxtCnrPreAndPost}
\end{figure}

\begin{figure}[tbh!]
  \caption{\mymmfigcpt{\mycodefigcpt{doubleOneBeeperAtNxtCnr} \mymmfigcpt{မလုပ်ဆောင်မီ က) နှင့် လုပ်ဆောင်ပြီး ခ)}}}
  \begin{subfigure}[t]{0.48\textwidth}
      \adjincludegraphics[width=2.5in,trim={0 0 0 {.4\height}}, clip, left]{ch03/DoubleBeeperPile/doubleOneBeeperAtNxtCnrPre.jpg}
      \caption{}
      \label{fig:doubleOneBeeperAtNxtCnrPre}
  \end{subfigure}
  \hspace{0.1in}
  \begin{subfigure}[t]{0.48\textwidth}
    \adjincludegraphics[width=2.5in,trim={0 0 0 {.4\height}}, clip, left]{ch03/DoubleBeeperPile/doubleOneBeeperAtNxtCnrPost.jpg}
      \caption{}
      \label{fig:doubleOneBeeperAtNxtCnrPost}
  \end{subfigure}
  \label{fig:doubleOneBeeperAtNxtCnrPreAndPost}
\end{figure}

\begin{figure}[tbh!]
  \caption{\mymmfigcpt{\mmiteration လေးခု}}
  \begin{subfigure}[t]{0.48\textwidth}
      \adjincludegraphics[width=2.5in,trim={0 0 0 {.4\height}}, clip, left]{ch03/DoubleBeeperPile/getReadyForNxtPre1.jpg}
      \caption{}
      \label{fig:getReadyForNxtPre1}
  \end{subfigure}
  \hspace{0.1in}
  \begin{subfigure}[t]{0.48\textwidth}
    \adjincludegraphics[width=2.5in,trim={0 0 0 {.4\height}}, clip, left]{ch03/DoubleBeeperPile/getReadyForNxtPost1.jpg}
      \caption{}
      \label{fig:getReadyForNxtPost1}
  \end{subfigure}
  \begin{subfigure}[t]{0.48\textwidth}
    \adjincludegraphics[width=2.5in,trim={0 0 0 {.4\height}}, clip, left]{ch03/DoubleBeeperPile/getReadyForNxtPre2.jpg}
    \caption{}
    \label{fig:getReadyForNxtPre2}
  \end{subfigure}
  \hspace{0.1in}
  \begin{subfigure}[t]{0.48\textwidth}
    \adjincludegraphics[width=2.5in,trim={0 0 0 {.4\height}}, clip, left]{ch03/DoubleBeeperPile/getReadyForNxtPost2.jpg}
      \caption{}
      \label{fig:getReadyForNxtPost2}
  \end{subfigure}
  \label{fig:getReadyForNxtPreAndPost}
\end{figure}

\begin{lstcodelong}[caption=ပထမစမ်းကြည့်ပုံ, label={lst:DoubleBeeperPileFull}] 
public class DoubleBeeperPile extends stanford.karel.Karel {
        public void run() {
                goToBeeperPile();
                doubleTheBeeperPileAtNxtCnr();
                turnAround();
                moveTheBeeperPileAtNxtCnr();
                turnAround();
        }

        private void goToBeeperPile() {
                while(noBeepersPresent()) {
                        move();
                }
        }

        private void doubleTheBeeperPileAtNxtCnr() {
                while(beepersPresent()) {
                        doubleOneBeeperAtNxtCnr();
                        getReadyForNxt();
                }
                move();
        }

        private void moveTheBeeperPileAtNxtCnr() {
                while(beepersPresent()) {
                        moveOneBeeperAtNxtCnr();
                        getReadyForNxt();
                }
                move();
        }
        private void moveOneBeeperAtNxtCnr() {
                if(beepersPresent()) {
                        pickBeeper();
                }
                move();
                putBeeper();
        }
        
        private void doubleOneBeeperAtNxtCnr() {
                if(beepersPresent()) {
                        pickBeeper();
                }
                move();
                putBeeper();
                putBeeper();
        }

        private void getReadyForNxt() {
                turnAround();
                move();
                turnAround();
        }

        private void turnAround() {
                turnLeft();
                turnLeft();
        }
}
\end{lstcodelong}
\end{sloppypar}


\end{document}